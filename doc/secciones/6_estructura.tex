%%%%%%%%%%%%%%%%%%%%%%%%%%%%%%%%%%%%%%
%% Estructura 
%%%%%%%%%%%%%%%%%%%%%%%%%%%%%%%

\section{Estructura }

Los ficheros que se entregan son los siguientes: 
\begin{itemize}
    \item \textbf{Instancias APC}: Ficheros con datos.
    \item \textbf{algoritmos búsqueda}: Generar vecinos. 
    \item \textbf{learner}: Abstrae algoritmos de búsqueda local.
    \item \textbf{resultados}: Carpeta con csv de los resultados y los ficheros que los generan.
    \item \textbf{utils}: Funciones auxiliares útiles.
\end{itemize}

Para ejecutarlo basta con ejecutar \texttt{make result}.

Además todos los algoritmos se han ejecutado con semilla $0$. 

\section{Metodología}  

Se ha desarrollado la práctica con una metodología ágil que puede seguirse 
en el repositorio de github \url{https://github.com/BlancaCC/Problema-aprendizaje-pesos}.

El lenguaje utilizado ha sido Julia y además se han hecho implementaciones con código en paralelo 
(el cual no se contabiliza en las mediciones).
