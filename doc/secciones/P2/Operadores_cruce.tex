%%%%%%%%%%%%%%%%%%%
%% Operadores de cruce 
%%%%%%%%%%%%%%%%%

\section{Operadores de cruce}  

\subsection{ Operador de cruce BLX$-\alpha$}  

El pseudo código del operador de cruce creado es 

% Operador de cruce 
\begin{algorithm}[H]
    \caption{Operador de cruce BLX$-\alpha$}
    \hspace*{\algorithmicindent} 

        \textbf{Entrada}:
        \begin{itemize}
          \item $C_1 = (c_{1 1}, \ldots, c_{1 n}), C_2=(c_{2 1}, \ldots, c_{2 n})$ 
          dos vectores de dimensión $n$, que son los cromosomas. 
          \item  $\alpha \in [0,1]$.
        \end{itemize}
        
        \hspace*{\algorithmicindent} 

        \textbf{Salida}:
        Dos vectores $H_1, H_2$ de tamaño $n$ que son los descendientes.

    \begin{algorithmic}[1]
        % Para cada uno de los vectores
        \For{ $i \in \{1, \ldots, n\}$}
              \begin{align*}
                & c_{\max} \gets \max\{ c_{1 i}, c_{2 i}\} \\
                & c_{\min} \gets \min\{ c_{1 i}, c_{2 i}\} \\
                & I \gets c_{\max} - c_{\min} 
              \end{align*}
          
              \For{$k \in \{1,2\}$}
              $$h_{k j} \gets
              \text{ valor aleatorio perteneciente a } 
             [c_{min} - I \alpha, c_{max} + I \alpha]$$
           \EndFor 
           \EndFor 
          \For{$k \in \{1,2\}$}
             $$H_k \gets (h_{k 1}, \ldots, h_{k n}) $$
          \EndFor 
          
       \State \textbf{return} $H_1, H_2$
    \end{algorithmic}
  \end{algorithm}

