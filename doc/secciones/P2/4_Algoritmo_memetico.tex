%%%%%%%%%%%%%%%%%%%%%%%%%%%%%%%%%%%%%%%%%%%%%
% Consideraciones sobre el algoritmo memético  
%%%%%%%%%%%%%%%%%%%%%%%%%%%%%%%%%%%%%%%%%%%%%

\section{Algoritmo memético}  

Para el algoritmo genético generacional el mejor operador de cruce obtenido ha sido el \textit{BLX}, así que será el que utilicemos para este caso. 

\subsection{Descripción de la implementación}  

La idea que subyace es sencilla, se ha tomado como base la estructura del algoritmo genético generacional,
 y cuando se encuentra ante cierta generación dado un subconjunto 
 de cromosomas, estos se sustituyen por el resultado de hacer 
 búsqueda local sobre él. 

El criterio seleccionado para aplicar búsqueda local ha sido cada diez generaciones.

Así pues el pseudo código del algoritmo memético, sin profundizar  en los pasos 1 a 4 porque son idénticos a los algoritmos genéticos y ya explicada de los algoritmos genéticos sería: 

\textbf{ Entrada:} 
\begin{itemize}
    \item Las mismas que para el algoritmo genético generacional.
    \item \textit{num generaciones aplicar BL}: Cada cuántas generaciones se aplica búsqueda local. 
    \item \textit{ Criterio Selección  Cromosomas }: Una función que devuelve los cromosomas (a nivel de implementación índices) de los cromosomas a los que se le aplicará la búsqueda local. En la siguiente sección se profundizará sobre ellos. 
\end{itemize}

\textbf{Salida:} La misma que los algoritmos genético, el mejor vector de pesos encontrado.  

\begin{algorithm}[H]
    \caption{Algoritmo memético}
    \begin{algorithmic}[1]
        \State Se general la primera generación. 
        \State \textit{numero generación} $\gets$ $1$
        \State La construcción de las distintas generaciones viene dada por: 
        \While{$evaluaciones <$ \textit{evaluaciones máxima función evaluación}}
        \State \textbf{Paso 1: Selección}
        \State \textbf{Paso 2: Cruce }
        \State \textbf{Paso 3:  Mutación}
        \State \textbf{Paso 4:  Reemplazo}
        \State \textit{numero generación} $\gets$ \textit{numero generación} $+1$
        \State \textbf{Paso 5: Aplicar BL si se dan las condiciones} \\
        \Comment{Se comienza a describir el paso 5}
        \If{ \textit{numero generación}  $\cong 0  \mod $ \textit{num generaciones aplicar BL}  }
            \State Seleccionamos conjunto de cromosomas a los que aplicar búsqueda local: 

            \begin{equation*}
                \textit{CROMOSOMAS-BL}  \gets  \textit{ Criterio Selección  Cromosomas ( Cromosomas ) }
            \end{equation*}

            \State\textit{Cromosomas } $\gets$ \textit{Cromosomas } $\setminus$ \textit{CROMOSOMAS-BL}
            \Comment{Eliminamos los cromosomas que vamos a actualizar}
            \For{ $c \in$ \textit{CROMOSOMAS-BL} }
            \State \textit{resultado búsqueda local } $\gets$ Aplicar búsqueda local a $c$.
            \State\textit{Cromosomas } $\gets$ \textit{Cromosomas } $\cup$ $\{$\textit{resultado búsqueda local } $\}$
            \EndFor
        \EndIf 
        \EndWhile
        
    \end{algorithmic}
    
\end{algorithm}

Notemos que de introducir el otro antes se estaría introduciendo 
ruido

\subsection{ Criterios de selección de cromosomas } 

Estos criterios son los que determina el tipo de algoritmo memético 
frente al que nos encontramos. Hemos abstraído la implementación a dos tipos de memético según el subconjunto de datos al que se le aplique la búsqueda local. 

\subsubsection{Porcentaje aleatorio}  

Se recibe como argumentos
\begin{itemize}
    \item \textit{coeficiente selección}: Un coeficiente entre $(0,1]$ de tamaño de subconjunto.
    \item \textit {tamaño}: Rango superior de los índices con los que trabajar. 
    \item Parámetro auxiliar, para mantener la coherencia en los algoritmos meméticos y que no se utiliza. 
\end{itemize}

\textbf{Salida} Conjunto de índices (desordenados) y únicos a los que se le aplicará el algoritmo de búsqueda local. 
\begin{algorithm}[H]
    \caption{Algoritmo de selección de índice del tipo (tamaño población, generación)}
    \begin{algorithmic}[1]
        \State \textit{cantidad índices} $\gets$ redondeo( \textit{coeficiente selección} $\times$   \textit {tamaño})
        \State  \textit{índices} $\gets$ Generar de manera aleatoria \textit{cantidad índices} índices. \\
        (Estos índices no se repiten y toman valores enteros entre 1 y el tamaño población).  
        \State \textbf{return}  \textit{índices}
    \end{algorithmic}
\end{algorithm}

\subsubsection{Subconjunto de los mejores cromosomas}  

Ahora el subconjunto seleccionado es el de los mejores cromosomas, luego la implementación es la siguiente : 

\textbf{Entrada}: 
\begin{itemize}
    \item \textit{coeficiente selección}: Un coeficiente entre $(0,1]$ de tamaño de subconjunto.
    \item \textit {tamaño}: Rango superior de los índices con los que trabajar. 
    \item \textit{valores función evaluación}: lista con el valor del \textit{fitness} del cromosoma i-ésimo en la i-ésima posición. 
\end{itemize}

\begin{algorithm}[H]
    \caption{Algoritmo de selección de índice del tipo (tamaño población, generación)}
    \begin{algorithmic}[1]
        \State \textit{cantidad índices} $\gets$ redondeo( \textit{coeficiente selección} $\times$   \textit {tamaño})
        \For{$i \in \{1, \ldots, \textit{cantidad índices} \}$}
            \State \textit{índice del mejor} $\gets$ obtener índice del mejor valor de \textit{valores función evaluación}. 
            \State \textit{índices}[$i$] $\gets$  \textit{índice del mejor}. 
            \State \textit{valores función evaluación}[\textit{índice del mejor}] $\gets - \infty$
            \Comment{Cambiamos su valor de evaluación para que no se vuelva a seleccionar}

        \EndFor
        \State\textbf{return}  \textit{índices}
    \end{algorithmic}
\end{algorithm}