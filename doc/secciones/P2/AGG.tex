%%%%%%%%%%%%%%%%%%%%%%%%%%%%%%%%%%%%%%%%%%%%%%%%%%%%%%%%
%% Algoritmos genéticos  generalistas 
%%
%%%%%%%%%%%%%%%%%%%%%%%%%%%%%%%%%%%%%%%%%%%%%%%%%%%%%%%%%

\section{ Algoritmos genéticos}
\subsection{Algoritmos genéticos generalistas}  
Para realizar esta solución hemos planteado el siguiente esquema. 

\begin{enumerate}
    \item Inicializamos la primera generación $t=1$ con $M = 30$ cromosomas que son
    vectores inicializado de manera aleatoria en $\{0\} \cup [0.1; 1]$ de tamaño $d$  donde $d$ es el número de atributos a analizar.  
    \item  Evaluamos la función objetivo de estos resultados. 
    \item Procedemos a realizar las sucesivas generaciones hasta que se cumpla el criterio de parada. 
\end{enumerate}

La generación de las sucesivas generaciones consiste en el siguientes algoritmo que tiene de  
\textbf{Entrada}:
        \begin{itemize}
          \item \texttt{EMFE}: evaluaciones máximas función evaluación.
          \item $M \in \mathbb{N}$: número de cromosomas en cada generación. 
          \item $P_c \in [0,1]$: probabilidad de cruce. 
          \item $P_m \in [0,1]$: probabilidad de mutación. 
          \item $d$ tamaño de cada cromosoma. 
          \item $F$ función de evaluación. 
          \item \textit{Función-cruce}$: C \times C \longrightarrow C \times C$ Algoritmo de cruce. 
          \item \textit{Función-Mutación}$: C \longrightarrow C $ Función de mutación. 
        \end{itemize}

y de 
\textbf{Salida}:Última generación de cromosomas a la que se ha llegado,
        consiste en un conjunto de tamaño $M$ de pares de cromosomas y su valor en la función de evaluación.

% Algoritmo genético generacional  
\begin{algorithm}[H]
    \caption{Algoritmo genético generacional}

    \begin{algorithmic}[1]
        \State Generamos primera generación \\
        \begin{align*}
            \textit{Generación }\gets & \text{ conjunto de tamaño $M$ tal que } \\
            \{ \quad  & \\
                &(v, f(v)) | v \gets \text{vector  aleatorio con valores en } [0,1] \\ 
                & \text{ donde si  los valores menores que } 0.1 \text{ se transforman a } 0. \\
            \}. \quad &
        \end{align*}
        % Cálculos auxiliares 
        \State Cálculo de valores auxiliares.
        \begin{itemize}
            \item \textit{numero de cruces}  $\gets round\left(P_c \frac{M}{2}\right).$
            \item \textit{parejas a cruzar} $\gets$ vector de tamaño del \textit{numero de cruces} con valores aleatorios y únicos de enteros entre $1$ y $\frac{M}{2}$.
            \item \textit{índices cruce } $\gets \{ (2 i-2, 2 i -1 ) | i \in \textit{ parejas a cruzar } \}$. 
            \item \textit{cantidad de cromosomas a mutar }  $\gets round(P_m M)$.
            \item \textit{ índices a mutar } $\gets$ vector de índices con valores entre 1 y $M$ de tamaño \textit{ la cantidad cromosomas a mutar}. 
        \end{itemize}

        \textit{evaluaciones} $\gets M$.
        % Procedemos a evolucionar 
        \While{$evaluaciones < EMFE$}
            % Selección
            \State \textbf{Paso 1: Selección}
            \State \textit{Seleccionados} $\gets$ conjunto formado tras $M$ enfrentamientos binarios binarios de \textit{Generación }. 
            % Cruce 
            \State \textbf{Paso 2: Cruce }
            \For{ $(i,j) \in$\textit{índices cruce }  }
            \begin{align*}
                & h_1, h_2 \gets \textit{Función- cruce}(Seleccionados[i], Seleccionados[j]) \\
                & Seleccionados[i] \gets h_1 \\
                & Seleccionados[j] \gets h_2 
            \end{align*}
            \EndFor

            % Mutación 
            \State \textbf{Paso 3:  Mutación}
            \For{ $i \in$ \textit{ índices a mutar }  }
            \begin{align*}
                Seleccionados[i] \gets \textit{Función-mutación}(Seleccionados[i]).
            \end{align*}
            \EndFor
        \EndWhile
        % Reemplazo 
        \State \textbf{Paso 4:  Reemplazo}
        \Comment{Toda la generación es reemplazada}
        \begin{equation*}
            \textit{Generación } \gets 
            \{
                (s, F(s)) | s \in  Seleccionados
            \}.
        \end{equation*}
        % Actualizamos valores
        \State $evaluaciones \gets evaluaciones + M$ \Comment{Número evaluaciones totales función evaluación }

        % Para cada uno de los vectores          
       \State \textbf{return} \textit{Generación }.
    \end{algorithmic}
  \end{algorithm}

TODO: 
  Notemos que falta por explicar el algoritmo de torneo binario 
  y como función de mutación se mutación se ha usado la de general vecinos de la práctica anterior TODO:  añadir referencia. 
