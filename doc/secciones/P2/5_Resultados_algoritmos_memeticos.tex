%%%%%%%%%%%%%%%%%%%%%%%%%%%%%%%%%%%%%
% Resultados de los experimentos realizados en los algoritmos meméticos
%%%%%%%%%%%%%%%%%%%%%%%%%%%%%%%%%%%%%

\section{Resultados algoritmo memético}

% Memético 10 1
\subsection{Memético (10,1)}
\subsubsection{Ionosphere}
\begin{table}[H]
    \centering
    \caption{Resultados algoritmo Memético (10,1) para Ionosphere }
    \begin{tabular}{|l|l|l|l|l|}
    \hline
        Nombre\_Fila & Clasificación & Reducción & Agregación & Tiempo \\ \hline
        Partición 1 & 84,507 & 52,941 & 68,724 & 337,175 \\ \hline
        Partición 2 & 91,429 & 38,235 & 64,832 & 325,153 \\ \hline
        Partición 3 & 82,857 & 67,647 & 75,252 & 736,040 \\ \hline
        Partición 4 & 85,714 & 44,118 & 64,916 & 337,851 \\ \hline
        Partición 5 & 87,143 & 55,882 & 71,513 & 323,638 \\ \hline
        Medias  & 86,330 & 51,765 & 69,047 & 411,971 \\ \hline
        Desviación típica & 3,257 & 11,315 & 4,459 & 181,279 \\ \hline
    \end{tabular}
    \label{AM-10-1-Iosphere}
\end{table}

\subsubsection{Parkinson}
\begin{table}[H]
    \centering
    \caption{Resultados algoritmo Memético (10,1) para Parkinson}
    \begin{tabular}{|l|l|l|l|l|}
    \hline
        Nombre\_Fila & Clasificación & Reducción & Agregación & Tiempo \\ \hline
        Partición 1 & 94,872 & 68,182 & 81,527 & 88,284 \\ \hline
        Partición 2 & 100,000 & 59,091 & 79,545 & 101,643 \\ \hline
        Partición 3 & 89,744 & 72,727 & 81,235 & 75,896 \\ \hline
        Partición 4 & 92,308 & 54,545 & 73,427 & 88,233 \\ \hline
        Partición 5 & 89,744 & 63,636 & 76,690 & 102,489 \\ \hline
        Medias  & 93,333 & 63,636 & 78,485 & 91,309 \\ \hline
        Desviación típica & 4,291 & 7,187 & 3,419 & 11,045 \\ \hline
    \end{tabular}
    \label{AM-10-1-Parkinson}
\end{table}

\subsubsection{Spectf Heart}
\begin{table}[H]
    \centering
    \caption{Resultados algoritmo Memético (10,1) para Spectf Heart }
    \begin{tabular}{|l|l|l|l|l|}
    \hline
        Nombre\_Fila & Clasificación & Reducción & Agregación & Tiempo \\ \hline
        Partición 1 & 80,000 & 52,273 & 66,136 & 554,077 \\ \hline
        Partición 2 & 77,143 & 43,182 & 60,162 & 902,334 \\ \hline
        Partición 3 & 74,286 & 40,909 & 57,597 & 480,201 \\ \hline
        Partición 4 & 90,000 & 52,273 & 71,136 & 548,866 \\ \hline
        Partición 5 & 82,609 & 40,909 & 61,759 & 906,538 \\ \hline
        Medias  & 80,807 & 45,909 & 63,358 & 678,403 \\ \hline
        Desviación típica & 6,008 & 5,883 & 5,343 & 208,394 \\ \hline
    \end{tabular}
    \label{AM-10-1-Spectf-Heart}
\end{table}


% Memético 10 1
\subsection{Memético (10,0.1)}
\subsubsection{Ionosphere}
\begin{table}[H]
    \centering
    \caption{Resultados algoritmo Memético (10,0.1) para Ionosphere }
    \begin{tabular}{|l|l|l|l|l|}
    \hline
        Nombre\_Fila & Clasificación & Reducción & Agregación & Tiempo \\ \hline
        Partición 1 & 88,732 & 64,706 & 76,719 & 341,904 \\ \hline
        Partición 2 & 87,143 & 61,765 & 74,454 & 272,115 \\ \hline
        Partición 3 & 84,286 & 47,059 & 65,672 & 225,594 \\ \hline
        Partición 4 & 82,857 & 52,941 & 67,899 & 339,055 \\ \hline
        Partición 5 & 91,429 & 52,941 & 72,185 & 273,106 \\ \hline
        Medias  & 86,889 & 55,882 & 71,386 & 290,355 \\ \hline
        Desviación típica & 3,432 & 7,204 & 4,563 & 49,632 \\ \hline
    \end{tabular}
    \label{AM-10-01-Ionosphere}
\end{table}

\subsubsection{Parkinson}
\begin{table}[H]
    \centering
    \caption{Resultados algoritmo Memético (10;0,1) para  Parkinson}
    \begin{tabular}{|l|l|l|l|l|}
    \hline
        Nombre\_Fila & Clasificación & Reducción & Agregación & Tiempo \\ \hline
        Partición 1 & 87,179 & 81,818 & 84,499 & 91,641 \\ \hline
        Partición 2 & 89,744 & 68,182 & 78,963 & 110,445 \\ \hline
        Partición 3 & 89,744 & 77,273 & 83,508 & 83,708 \\ \hline
        Partición 4 & 92,308 & 59,091 & 75,699 & 91,584 \\ \hline
        Partición 5 & 84,615 & 77,273 & 80,944 & 110,660 \\ \hline
        Medias  & 88,718 & 72,727 & 80,723 & 97,608 \\ \hline
        Desviación típica & 2,924 & 9,091 & 3,550 & 12,250 \\ \hline
    \end{tabular}
    \label{AM-10-01-Parkinson}
\end{table}

\subsubsection{Spectf Heart}
\begin{table}[H]
    \centering
    \caption{Resultados algoritmo Memético (10;0,1) para  Spectf Heart}
    \begin{tabular}{|l|l|l|l|l|}
    \hline
        Nombre\_Fila & Clasificación & Reducción & Agregación & Tiempo \\ \hline
        Partición 1 & 85,714 & 52,273 & 68,994 & 452,539 \\ \hline
        Partición 2 & 78,571 & 50,000 & 64,286 & 364,241 \\ \hline
        Partición 3 & 84,286 & 54,545 & 69,416 & 303,492 \\ \hline
        Partición 4 & 94,286 & 50,000 & 72,143 & 444,916 \\ \hline
        Partición 5 & 85,507 & 47,727 & 66,617 & 366,211 \\ \hline
        Medias  & 85,673 & 50,909 & 68,291 & 386,280 \\ \hline
        Desviación típica & 5,625 & 2,591 & 2,977 & 62,392 \\ \hline
    \end{tabular}
    \label{AM-10-01-Spectf-Heart}
\end{table}

% Memético 10 1
\subsection{Memético (10,0.1 mejores) }

\subsubsection{Ionosphere}
\begin{table}[H]
    \centering
    \caption{Resultados algoritmo Memético (10;0,1 mejores) para Ionosphere}
    \begin{tabular}{|l|l|l|l|l|}
    \hline
        Nombre\_Fila & Clasificación & Reducción & Agregación & Tiempo \\ \hline
        Partición 1 & 87,324 & 61,765 & 74,544 & 294,436 \\ \hline
        Partición 2 & 91,429 & 58,824 & 75,126 & 247,371 \\ \hline
        Partición 3 & 85,714 & 50,000 & 67,857 & 184,478 \\ \hline
        Partición 4 & 87,143 & 61,765 & 74,454 & 293,941 \\ \hline
        Partición 5 & 90,000 & 58,824 & 74,412 & 246,153 \\ \hline
        Medias  & 88,322 & 58,235 & 73,279 & 253,276 \\ \hline
        Desviación típica & 2,327 & 4,833 & 3,044 & 45,184 \\ \hline
    \end{tabular}
    \label{AM-10-01mej-Ionosphere}
\end{table}

\subsubsection{Parkinson}
\begin{table}[H]
    \centering
    \caption{Resultados algoritmo Memético (10;0,1 mejores) para Parkinson}
    \begin{tabular}{|l|l|l|l|l|}
    \hline
        Nombre\_Fila & Clasificación & Reducción & Agregación & Tiempo \\ \hline
        Partición 1 & 84,615 & 77,273 & 80,944 & 58,288 \\ \hline
        Partición 2 & 97,436 & 50,000 & 73,718 & 64,764 \\ \hline
        Partición 3 & 92,308 & 68,182 & 80,245 & 73,315 \\ \hline
        Partición 4 & 92,308 & 77,273 & 84,790 & 72,157 \\ \hline
        Partición 5 & 92,308 & 72,727 & 82,517 & 73,636 \\ \hline
        Medias  & 91,795 & 69,091 & 80,443 & 68,432 \\ \hline
        Desviación típica & 4,587 & 11,318 & 4,145 & 6,729 \\ \hline
    \end{tabular}
    \label{AM-10-01mej-Parkinson}
\end{table}

\subsubsection{Spectf Heart}

\begin{table}[H]
    \centering
    \caption{Resultados algoritmo Memético (10;0,1 mejores) para Spectf Heart }
        \begin{tabular}{|l|l|l|l|l|}
            \hline
                Nombre\_Fila & Clasificación & Reducción & Agregación & Tiempo \\ \hline
                Partición 1 & 84,286 & 50,000 & 67,143 & 535,577 \\ \hline
                Partición 2 & 80,000 & 52,273 & 66,136 & 370,210 \\ \hline
                Partición 3 & 81,429 & 54,545 & 67,987 & 290,493 \\ \hline
                Partición 4 & 92,857 & 54,545 & 73,701 & 523,779 \\ \hline
                Partición 5 & 84,058 & 68,182 & 76,120 & 374,544 \\ \hline
                Medias  & 84,526 & 55,909 & 70,217 & 418,921 \\ \hline
                Desviación típica & 4,994 & 7,115 & 4,418 & 106,583 \\ \hline
            \end{tabular}
            \label{AM-10-01mej-Spectf-Heart}
\end{table}

\section{Análisis de los resultados obtenido para meméticos}
Quedan recogidos los resultado sobre algoritmos meméticos en la tabla 
\ref{Tabla:comparativas-algoritmos-geneticos}. 

\begin{table}[H]
    \centering
    \resizebox{\columnwidth}{!}{%
      \begin{tabular}{|c|c|c|c|c|c|c|c|c|c|c|c|c| }
     \hline
     & \multicolumn{4}{|c|}{\textit{Ionosphere}} 
     & \multicolumn{4}{|c|}{\textit{Parkinsons}} 
     &\multicolumn{4}{|c|}{\textit{Spectf-heart}}\\
     \hline
       Algoritmo
       & $\%\_$clas & $\%\_$red & Agr. & T (s) 
       & $\%\_$clas & $\%\_$red & Agr. & T (s)  
       & $\%\_$clas & $\%\_$red & Agr. & T (s) 
       \\ \hline
    AM-(10,1)
    % class | redu | agre | tiempo 
    & 86,330 & 51,765 & 69,047 & 411,971 % Ionosphere
    & 93,333 & 63,636 & 78,485 & 91,309 %Parkinson
    & 80,807 & 45,909 & 63,358 & 678,403  % Spectf- heart
    \\ \hline

    AM-(10,0.1)
    % class | redu | agre | tiempo 
    & 86,889 & 55,882 & 71,386 & 290,355 % Ionosphere
    & 88,718 & 72,727 & 80,723 & 97,608 %Parkinson
    & 85,673 & 50,909 & 68,291 & 386,280   % Spectf- heart
    \\ \hline

    AM-(10,0.1 mejores)
    % class | redu | agre | tiempo 
    & 88,322 & 58,235 & 73,279 & 253,276 % Ionosphere
    & 91,795 & 69,091 & 80,443 & 68,432  %Parkinson
    & 84,526 & 55,909 & 70,217 & 418,921  % Spectf- heart
    \\ \hline
    \end{tabular}
    }
    \caption{Comparativas Métricas algoritmos genéticos}
    \label{Tabla:comparativas-algoritmos-memeticos}
\end{table}

Como era de esperar los mejores resultados, independientemente de la base de datos escogida han sido (por orden de mejor a peor):

\begin{enumerate}
    \item AM-(10,0.1 mejores).
    \item AM-(10,0.1).
    \item AM-(10,1). 
\end{enumerate}

El  motivo de esto es sencillo, búsqueda local lo que hace es dado un 
cromosoma \textit{explorarlo} para mejorarlo; el problema de esto es que en 
nuestro algoritmo estamos limitando el número de llamadas a la función de 
activación, entonces se produce el siguiente efecto: \\

AM-(10,1) al explorar todas las soluciones, si bien avanza de una manera 
más \textit{robusta}, es decir, explorando todos los cromosomas; consume, la 
inversa del porcentaje de búsqueda local de los otros veces, más de 
evaluaciones de la función de activación en la misma busca. Esto produce que se exploren menos generaciones. 
\\

Para compensar este efecto, puede tomarse un subconjunto de los cromosomas, la heurística más prometedora consiste en explorar aquellos que tienen más potencial por el valor de su función de activación. 
\\

Como vemos estas reflexiones son apoyadas por los resultado empíricos obtenidos, lo cual no quiere decir que no exista un conjunto de datos para el cual se de la situación contraria. 

\section{Reflexión: Relevancia de dónde aplicar la búsqueda local}  

Obsérvese que el pseudo código aportado el algoritmo de búsqueda local se ha 
introducido en el paso 5. Tiene gran relevancia su situación, ya que la evolución genética, a pesar de conservar el elitismo de un cromosoma, introduce variaciones aleatorias que nos pueden separar de la solución. 
Por lo tanto, finalizar la generación con el algoritmo de búsqueda local 
dará por lo general mejores resultados que hacerlo al comienzo de la la 
expansión de la generación, es decir, en un hipotético paso 0. 

Como sustento experimental hemos recreado tal situación, ahora la búsqueda local se haría en: 

% Añade pseudocódigo 

\begin{algorithm}[H]
    \caption{Algoritmo memético, con búsqueda local en paso 0}
    \begin{algorithmic}[1]
        \State Se general la primera generación. 
        \State \textit{numero generación} $\gets$ $1$
        \State La construcción de las distintas generaciones viene dada por: 
        \While{$evaluaciones <$ \textit{evaluaciones máxima función evaluación}}
        \State \textbf{Paso 0: Aplicar BL si se dan las condiciones} \\
        \Comment{Se comienza a describir el paso 0}
        \If{ \textit{numero generación}  $\cong 0  \mod $ \textit{num generaciones aplicar BL}  }
            \State Seleccionamos conjunto de cromosomas a los que aplicar búsqueda local: 

            \begin{equation*}
                \textit{CROMOSOMAS-BL}  \gets  \textit{ Criterio Selección  Cromosomas ( Cromosomas ) }
            \end{equation*}

            \State\textit{Cromosomas } $\gets$ \textit{Cromosomas } $\setminus$ \textit{CROMOSOMAS-BL}
            \Comment{Eliminamos los cromosomas que vamos a actualizar}
            \For{ $c \in$ \textit{CROMOSOMAS-BL} }
            \State \textit{resultado búsqueda local } $\gets$ Aplicar búsqueda local a $c$.
            \State\textit{Cromosomas } $\gets$ \textit{Cromosomas } $\cup$ $\{$\textit{resultado búsqueda local } $\}$
            \EndFor
        \EndIf 
        \State \textbf{Paso 1: Selección}
        \State \textbf{Paso 2: Cruce }
        \State \textbf{Paso 3:  Mutación}
        \State \textbf{Paso 4:  Reemplazo}
        \State \textit{numero generación} $\gets$ \textit{numero generación} $+1$
  
        \EndWhile
    \end{algorithmic}
\end{algorithm}


Para la cual hemos obtenido los siguientes resultados medios.

\begin{table}[H]
    \centering
    \resizebox{\columnwidth}{!}{%
      \begin{tabular}{|c|c|c|c|c|c|c|c|c|c|c|c|c| }
     \hline
     & \multicolumn{4}{|c|}{\textit{Ionosphere}} 
     & \multicolumn{4}{|c|}{\textit{Parkinsons}} 
     &\multicolumn{4}{|c|}{\textit{Spectf-heart}}\\
     \hline
       Algoritmo
       & $\%\_$clas & $\%\_$red & Agr. & T (s) 
       & $\%\_$clas & $\%\_$red & Agr. & T (s)  
       & $\%\_$clas & $\%\_$red & Agr. & T (s) 
       \\ \hline
    AM-(10,1) BL en Paso 5
    % class | redu | agre | tiempo 
    & 86,330 & 51,765 & 69,047 & 411,971 % Ionosphere
    & 93,333 & 63,636 & 78,485 & 91,309 %Parkinson
    & 80,807 & 45,909 & 63,358 & 678,403  % Spectf- heart
    \\ \hline
    AM-(10,1) BL en paso 0
    % class | redu | agre | tiempo 
    & 86.901 & 47.647 & 69,047 & 333.633 % Ionosphere
    & 89.230 & 61.818 & 75.524& 87.341 %Parkinson
    & 82.223 & 48.636 & 65.429 & 437.089  % Spectf- heart
    \\ \hline


    \end{tabular}
    }
    \caption{Comparativa en AM-(10,1) en dónde se posiciona la búsqueda local}
    \label{Tabla:comparativas-algoritmos-AM-cambiandPAso}
\end{table}

Los resultados ponen de manifiesto que se trata de una heurística, que por lo general que mejoraría los resultado,
pero existen situaciones como la última en los que no. 

Además inducen a generar una nueva modificación a esta serie
de algoritmos, que consistirían en realizar una búsqueda 
local en la última generación, es decir volver a aplicar el 
paso 5 tras abandonar el bucle while, bastaría con disminuir 
el número de generaciones que se crean para compensar el número de evaluaciones del \textit{fitness}. 

Observemos además que en las comparativas, en el caso de paso 0, se ha tenido mejor clasificación en el primero, esto solo ha sido producto de la casualidad, la métrica en la que debemos de fijarnos debe de ser siempre la agregación. 








