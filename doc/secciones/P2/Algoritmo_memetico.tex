%%%%%%%%%%%%%%%%%%%%%%%%%%%%%%%%%%%%%%%%%%%%%
% Consideraciones sobre el algoritmo memético  
%%%%%%%%%%%%%%%%%%%%%%%%%%%%%%%%%%%%%%%%%%%%%

\section{Algoritmo memético}  

Para el algoritmo genético generacional el mejor operador de cruce obtenido ha sido el \textit{BLX}, así que será el que utilicemos para este caso. 

\subsection{Descripción de la implementación}  

La idea que subyace es sencilla, se ha tomado como base la estructura del algoritmo genético generacional,
 y cuando se encuentra ante cierta generación dado un subconjunto 
 de cromosomas, estos se sustituyen por el resultado de hacer 
 búsqueda local sobre él. 

El criterio seleccionado para aplicar búsqueda local ha sido cada diez generaciones.

Así pues el pseudo código del algoritmo memético, sin profundizar  en los pasos 1 a 4 porque son idénticos a los algoritmos genéticos y ya explicada de los algoritmos genéticos sería: 

\begin{algorithm}[H]
    \caption{Algoritmo memético}
    \begin{algorithmic}[1]
        \State Se general la primera generación. 
        \State La construcción de las distintas generaciones viene dada por: 
        \While{$evaluaciones < EMFE$}
        \State \textbf{Paso 1: Selección}
        \State \textbf{Paso 2: Cruce }
        \State \textbf{Paso 3:  Mutación}
        \State \textbf{Paso 4:  Reemplazo}
        \State \textit{numero generación} $\gets$ \textit{numero generación} $+1$
        \State \textbf{Paso 5: Aplicar BL si se dan las condiciones} \\
        \Comment{Se cominza a describir el paso 5}
        \If{\textit{num generaciones aplicar BL}    $\cong \mod $ \textit{numero generación}  }
            \State Hola esto está dentro del if
        \EndIf 
        \EndWhile
        
    \end{algorithmic}
    
\end{algorithm}

Notemos que de introducir el otro antes se estaría introduciendo 
ruido
