%%%%%%%%%%%%%%%%%%%%%%%%%%%%%%%%%%%%%%%%%%%%%%%%%%%%%%
%           Resultados finales de la P2
%%%%%%%%%%%%%%%%%%%%%%%%%%%%%%%%%%%%%%%%%%%%%%%%%%%%%%

\section{Compendio de resultados obtenidos}


\begin{table}[H]
    \centering
    \resizebox{\columnwidth}{!}{%
      \begin{tabular}{|c|c|c|c|c|c|c|c|c|c|c|c|c| }
     \hline
     & \multicolumn{4}{|c|}{\textit{Ionosphere}} 
     & \multicolumn{4}{|c|}{\textit{Parkinsons}} 
     &\multicolumn{4}{|c|}{\textit{Spectf-heart}}\\
     \hline
       Algoritmo
       & $\%\_$clas & $\%\_$red & Agr. & T (s) 
       & $\%\_$clas & $\%\_$red & Agr. & T (s)  
       & $\%\_$clas & $\%\_$red & Agr. & T (s) 
       \\ \hline
    % Algoritmo práctica primera 
    1NN base
    & 86.044 &0 & 43.022 & 4.623 $\times 10^-3$
    & 95.897& 0 & 47.948   &2.312$\times 10^-3$ 
    & 82.522 & 0 & 41.261   & 4.109$\times 10^-3$ 
    \\ \hline

    Búsqueda local
    & 85.191 & 27.059 & 56.125  &  154.672  
    & 92.821 &  27.059  & 58.228 & 41.035
    & 82.807 &  20.455 & 51.631 & 357.893
    \\ \hline

    Greedy
    & 83.751 & 2.941 & 43.346& 49.051 $\times 10^-3$
    & 95.385 & 0 & 47.692 &  13.424$\times 10^-3$
    & 83.959 & 0&41.979 &56.008 $\times 10^-3$
    \\ \hline

    % Algoritmos genéticos
    AGG - BLX
    % class | redu | agre | tiempo 
    & 87,469 & 63,529& 75,499 & 307,142% Ionosphere
    & 88,717 & 76,363 & 82,540 & 94,256 %Parkinson
    & 82,795 & 64,091 & 73,443 & 450,690 % Spectf- heart
    \\ \hline

    AGG - Media 
    % class | redu | agre | tiempo 
    & 85,461 & 55,882 & 70,672 & 264,407 % Ionosphere
    & 89,744 & 69,091 & 79,417 & 72,522 %Parkinson
    & 83,383 & 51,818 & 67,601 & 432,108 % Spectf- heart
    \\ \hline

    AGE - BLX
    % class | redu | agre | tiempo 
    & 86,620 & 61,176 & 73,898 & 266,071  % Ionosphere
    & 88,205 & 71,818 & 80,012 & 91,395  %Parkinson
    & 82,530 & 50,909 & 66,720 & 295,179 % Spectf- heart
    \\ \hline

    AGE - Media 
    % class | redu | agre | tiempo 
    & 88,040 & 55,882 & 71,961 & 248,215  % Ionosphere
    & 90,256 & 68,182 & 79,219 & 82,772  %Parkinson
    & 83,391 & 52,727 & 68,059 & 267,132  % Spectf- heart
    \\ \hline
    % Algoritmos meméticos
    AM-(10,1)
    % class | redu | agre | tiempo 
    & 86,330 & 51,765 & 69,047 & 411,971 % Ionosphere
    & 93,333 & 63,636 & 78,485 & 91,309 %Parkinson
    & 80,807 & 45,909 & 63,358 & 678,403  % Spectf- heart
    \\ \hline

    AM-(10,0.1)
    % class | redu | agre | tiempo 
    & 86,889 & 55,882 & 71,386 & 290,355 % Ionosphere
    & 88,718 & 72,727 & 80,723 & 97,608 %Parkinson
    & 85,673 & 50,909 & 68,291 & 386,280   % Spectf- heart
    \\ \hline

    AM-(10,0.1 mejores)
    % class | redu | agre | tiempo 
    & 88,322 & 58,235 & 73,279 & 253,276 % Ionosphere
    & 91,795 & 69,091 & 80,443 & 68,432  %Parkinson
    & 84,526 & 55,909 & 70,217 & 418,921  % Spectf- heart
    \\ \hline


    \end{tabular}
    }
    \caption{Comparativas algoritmos capítulo 1 y capítulo 2}
    \label{Tabla:comparativas-algoritmos-P2}
\end{table}

Podemos observar que en términos globales el que mejor resultado ofrece es 
el algoritmo genético generacional con el operador de cruce BLX, sin embargo es además el que mayor tiempo de cálculo emplea. 
La segunda mejor opción, por el valor de agregación y que presenta una mejora importante a cuanto tiempo sería el memético considerando un subconjunto de mejores.   

Podemos además considerar una mejora sustancial de los resultados al introducir algoritmos genéticos y meméticos, que si bien han duplicado la agregación el tiempo también lo ha hecho, aunque en menor medida y con un crecimiento menor que lineal en relación con la agregación. 
\subsection{Reflexión sobre la relación entre reducción y clasificación}
Existe también una tendencia, en todos estos algoritmos  y particiones a que la tasa de clasificación siempre sea mucho mayor que la de reducción. Esto nos está indicado que al aumentar la reducción produce resultados de mayor perjuicio en clasificación, por lo que la agregación bajaría. 
