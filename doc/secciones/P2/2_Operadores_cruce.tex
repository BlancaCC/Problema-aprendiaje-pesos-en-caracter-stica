%%%%%%%%%%%%%%%%%%%
%% Operadores de cruce 
%%%%%%%%%%%%%%%%%

\section{Operadores de cruce}  

\subsection{ Operador de cruce BLX$-\alpha$}  

El pseudo código del operador de cruce creado es 

% Operador de cruce 
\begin{algorithm}[H]
    \caption{Operador de cruce BLX$-\alpha$}
    \hspace*{\algorithmicindent} 

        \textbf{Entrada}:
        \begin{itemize}
          \item $C_1 = (c_{1 1}, \ldots, c_{1 n}), C_2=(c_{2 1}, \ldots, c_{2 n})$ 
          dos vectores de dimensión $n$, que son los cromosomas. 
          \item  $\alpha \in [0,1]$.
        \end{itemize}
        
        \hspace*{\algorithmicindent} 

        \textbf{Salida}:
        Dos vectores $H_1, H_2$ de tamaño $n$ que son los descendientes.

    \begin{algorithmic}[1]
        % Para cada uno de los vectores
        \For{ $i \in \{1, \ldots, n\}$}
              \begin{align*}
                & c_{\max} \gets \max\{ c_{1 i}, c_{2 i}\} \\
                & c_{\min} \gets \min\{ c_{1 i}, c_{2 i}\} \\
                & I \gets c_{\max} - c_{\min} 
              \end{align*}
          
              \For{$k \in \{1,2\}$}
              $$h_{k j} \gets
              \text{ valor aleatorio perteneciente a } 
             [c_{min} - I \alpha, c_{max} + I \alpha]$$
           \EndFor 
           \EndFor 
          \For{$k \in \{1,2\}$}
             $$H_k \gets (h_{k 1}, \ldots, h_{k n}) $$
          \EndFor 
          
       \State \textbf{return} $H_1, H_2$
    \end{algorithmic}
  \end{algorithm}

\subsection{ Operador de cruce de media}   

Se pretende generar dos hijos $h_1, h_2$ a partir de dos cromosomas padres $c_1, c_2$, para ello 
dado un valor real $\alpha \in (0,1)$ aleatorio se generan los dos siguientes hijos: 
\begin{align*}
  h_1 &= \alpha c_1 + (1-\alpha ) c_2  \\
  h_2 &= \alpha c_2 + (1-\alpha ) c_1 
\end{align*}

Notemos que si $\alpha = 0.5$ entonces los hijos serán iguales. 

\subsection{Reflexión sobre los operadores de cruce}  

Cabría plantearse si existe un operador de cruce mejor que otro, 
para ello debemos deparar que 
los coeficientes del algoritmo por media siempre van a estar acotados por los coeficientes de los padres; mientras que para el de $BLX-\alpha$ no, 
ya que estarán dentro de cierto rango de confianza. 

Esta situación del operador de media genera una restricción implícita en la búsqueda de sucesores, la cual de manera teórica limita la capacidad de exploración de los datos y por ende podría afectar, por lo general a los resultados. 

De hecho a la vista de los resultados empíricos que veremos más adelante
esta teoría se refuerza.

Para corregir esta situación se podría considerar, dado un $\delta \in \mathbb{R}^+$  que $\alpha \in [-\delta, 1+\delta]$.   

Sin embargo esto mantendría el mismo factor variación para todos los coeficientes. 

Por otro lado el coste computacional de BLX es mayor, ya que para cada gen calcula un aleatorio y una serie de operaciones, el coste depende del tamaño del cromosoma. 

Ante estos dos paradigmas se podría buscar un compromiso entre 
coste computacional y resultados; realizando combinaciones lineales 
de subíndices concretos.  Es decir, supongamos que en vez de depender
del tamaño del cromosoma, se depende del natural $1 <r <<$\textit{tamaño cromosoma}  particiones aleatorias de los 
índices. 

Entonces podríamos construir a los hijos como:
Seleccionamos $\alpha_1, \ldots \alpha_r \in [-\delta, 1+\delta]$ aleatorios. 

los coeficientes del hijo $h$ vendrían dados como
\begin{equation*}
  h[j] = \alpha_i c_1  + (1-\alpha_i)c_2 \text{ con } j \in \textit{partición de índices}_i.
\end{equation*}