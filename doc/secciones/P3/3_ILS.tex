%%%%%%%%%%%%%%%%%%%%%%%%%%%%%%%%%%%%%%%%%%%%%%%%%%%%%%
% Búsqueda Local Reiterada 
%%%%%%%%%%%%%%%%%%%%%%%%%%%%%%%%%%%%%%%%%%%%%%%%%%%%%%

\section{Búsqueda Local Reiterada (ILS)}

Muy similar al algoritmo de búsqueda multiarranque 
básica, salvo que no se genera de nuevo 
el vector de inicio si no que se parte del mejor mutado
encontrado. 

El pseudo código resultante es el siguiente: 

\begin{algorithm}[H]
    \begin{algorithmic}
      \Procedure{Búsqueda Local Reiterada}{ \textit{numero iteraciones}}
      \State\textit{mejorW} $\gets Null$ 
      \State\textit{w} $\gets Null$ \Comment{Al ser null el algoritmo de búsqueda local crea un vector aleatorio}
      \State \textit{fitnessMejorW} $\gets - \infty$
      \For{$i \in \{1, \ldots, \textit{número operaciones}\}$}
        \State $w \gets $ Búsqueda Local (\textit{mejorW})
        \State $fitnessW \gets F(w)$
        \If{$fitnessW > fitnessMejorW$}
            \State $fitness-mejor-w \gets fitness-w$ 
            \State $mejorW \gets w$
        \EndIf
        \If{$fitnessW < fitnessMejorW$}
            \State $w \gets mejorW$ 
            \Comment{Asociamos el mejor para ahora mutarlo}
        \EndIf
        \State $w \gets $Mutación($w$)
        \EndFor
        \State \textbf{return } 
        \textit{mejorW}
      \EndProcedure
    \end{algorithmic}  
  \end{algorithm} 
El operador de mutación coincide con el utilizado 
en el algoritmo de generar vecinos de la práctica primera introduciendo como argumentos un
un cambio del $10\%$ de sus características
esto es: $t = 0.1 n$, además $\sigma = 0.4$.

Tiene el siguiente formato: 

\begin{algorithm}[H]
    \begin{algorithmic}
      \Procedure{Mutación}{ w, $\mu$, $\sigma$, n, coeficiente}
      \State Se genera \textit{index} una lista de los $round(n \times coeficiente)$ primeros elementos. 
      \For{ $i \in index $}
      \State $z \gets$ valor aleatorio en distribución normal de media $N(\mu, \sigma)$
      \State $w_i \gets \min(1, \max(w_i - z))$
      \EndFor
        \State \textbf{return } 
        \textit{mejorW}
      \EndProcedure
    \end{algorithmic}  
  \end{algorithm}

  Por como está diseñado nuestro algoritmo de búsqueda local, si con esta mutación se sale del rango se le aplicará el mapeo pertinente. 

El número de evaluaciones para cada BL será 
de 1000 evaluaciones. 

\subsection{Resultados}

\subsubsection{Parkinson}

\begin{table}[H]
    \centering
    \caption{Búsqueda Local Iterativa}
    \begin{tabular}{|l|l|l|l|l|}
    \hline
        Nombre Fila & Clasificación & Reducción & Agregación & Tiempo \\ \hline
        Partición 1 & 92,308 & 40,909 & 66,608 & 526,366 \\ \hline
        Partición 2 & 94,872 & 18,182 & 56,527 & 621,708 \\ \hline
        Partición 3 & 94,872 & 27,273 & 61,072 & 530,507 \\ \hline
        Partición 4 & 94,872 & 18,182 & 56,527 & 573,664 \\ \hline
        Partición 5 & 94,872 & 13,636 & 54,254 & 559,754 \\ \hline
        Medias  & 94,359 & 23,636 & 58,998 & 562,400 \\ \hline
        Desviación típica & 1,147 & 10,852 & 4,923 & 38,623 \\ \hline
    \end{tabular}
    \label{ILS-Parkinson}
\end{table}


Nota: Para las siguientes bases no he utilizado validación cruzada con 5 \textit{fold}. 
El motivo de esto es que los tiempos eran demasiado largos (muy similares a los de BMB como se puede ver en Parkinson, que sí ha sido ejecutado con validación cruzada). 

El dato que se va a mostrar se corresponde con 
entrenar con cuatro quintos de los datos y validar con un quinto. 


\subsubsection{Ionosphere}

\begin{table}[H]
    \centering
    \caption{Búsqueda Local Iterativa Ionosphere}
    \begin{tabular}{|l|l|l|l|l|}
    \hline
        Nombre Fila & Clasificación & Reducción & Agregación & Tiempo \\ \hline
        Partición 1  & 87,324 & 11,765 & 49,544 & 2660,953 \\ \hline
    \end{tabular}
    \label{ILS-Ionosphere}
\end{table}



\subsubsection{Spectf Heart}

\begin{table}[H]
    \centering
    \caption{Búsqueda Local Iterativa Spectf Heart}
    \begin{tabular}{|l|l|l|l|l|}
    \hline
        Nombre Fila & Clasificación & Reducción & Agregación & Tiempo \\ \hline
        Partición 1 & 84.2857  & 31.8182   &  58.0519  & 4456.77 
        \\ \hline
    \end{tabular}
    \label{ILS-Heart}
\end{table}

\subsection*{Conclusiones} 
Aplíquense las mismas conclusiones de BMB. 

