%%%%%%%%%%%%%%%%%%%%%%%%%%%%%%%%%%%%%%%%%%%%%%%%%%%%%%
%  Búsqueda multi arranque básica
%%%%%%%%%%%%%%%%%%%%%%%%%%%%%%%%%%%%%%%%%%%%%%%%%%%%%%

\section{Búsqueda Local Multiarranque}  

\subsection{Descripción }

Se generan 15 soluciones aleatorias
iniciales con el algoritmo de Búsqueda Local y optimizar cada una de ellas con el algoritmo
de Búsqueda Local. 
El pseudo código se encuentra detallado en la sección 
\ref{algoritmo-busqueda-local} (concretamente en el \ref{primero-mejor}), además puede observarse que de no introducirse como argumento el vector inicial se genera aleatoriamente, éste sistema será el que utilicemos para generar la lista de candidatos. 

Se devolverá la mejor solución encontrada en todo
el proceso. 

\subsection{Resultados}
\subsubsection{Ionosphere}
\begin{table}[!ht]
    \centering
    \caption{Búsqueda Local Multiarranque Ionosphere}
    \begin{tabular}{|l|l|l|l|l|}
    \hline
        Nombre Fila & Clasificación & Reducción & Agregación & Tiempo \\ \hline
        Partición 1 & 87,324 & 26,471 & 56,897 & 3279,619 \\ \hline
        Partición 2 & 92,857 & 23,529 & 58,193 & 3275,576 \\ \hline
        Partición 3 & 82,857 & 14,706 & 48,782 & 3023,867 \\ \hline
        Partición 4 & 81,429 & 29,412 & 55,420 & 3125,478 \\ \hline
        Partición 5 & 90,000 & 17,647 & 53,824 & 3053,495 \\ \hline
        Medias  & 86,893 & 22,353 & 54,623 & 3151,607 \\ \hline
        Desviación típica & 4,784 & 6,099 & 3,651 & 120,811 \\ \hline
    \end{tabular}
    \label{BLB-Ionosphere}
\end{table}
\subsubsection{Parkinson}
\begin{table}[!ht]
    \centering
    \caption{Búsqueda Local Multiarranque Parkinson}
    \begin{tabular}{|l|l|l|l|l|}
    \hline
        Nombre Fila & Clasificación & Reducción & Agregación & Tiempo \\ \hline
        Partición 1 & 89,744 & 40,909 & 65,326 & 562,730 \\ \hline
        Partición 2 & 97,436 & 27,273 & 62,354 & 634,176 \\ \hline
        Partición 3 & 92,308 & 22,727 & 57,517 & 565,581 \\ \hline
        Partición 4 & 92,308 & 22,727 & 57,517 & 561,459 \\ \hline
        Partición 5 & 89,744 & 18,182 & 53,963 & 677,873 \\ \hline
        Medias  & 92,308 & 26,364 & 59,336 & 600,364 \\ \hline
        Desviación típica & 3,140 & 8,743 & 4,486 & 53,129 \\ \hline
    \end{tabular}
    \label{BLB-Parkinson}
\end{table}


\subsubsection{Spectf Heart}

\begin{table}[!ht]
    \centering
    \caption{Búsqueda Local Multiarranque Spectf Heart}
    \begin{tabular}{|l|l|l|l|l|}
    \hline
        Nombre Fila & Clasificación & Reducción & Agregación & Tiempo \\ \hline
        Partición 1 & 80,000 & 29,545 & 54,773 & 4190,116 \\ \hline
        Partición 2 & 77,143 & 31,818 & 54,481 & 3910,688 \\ \hline
        Partición 3 & 84,286 & 27,273 & 55,779 & 3704,032 \\ \hline
        Partición 4 & 94,286 & 20,455 & 57,370 & 4161,981 \\ \hline
        Partición 5 & 79,710 & 20,455 & 50,082 & 3908,906 \\ \hline
        Medias  & 83,085 & 25,909 & 54,497 & 3975,145 \\ \hline
        Desviación típica & 6,766 & 5,232 & 2,714 & 201,968 \\ \hline
    \end{tabular}
    \label{BLB-hert}
\end{table}

\subsection{Conclusión}  

Los resultados en cuanto a tiempo bondad de la agregación 
son bastante malos, daremos una explicación de ésto en las conclusiones 
finales del proyecto; la idea intuitiva es 
que su potencial de encontrar soluciones buenas 
está limitado al de Búsqueda local. 

Es decir, si Búsqueda Local para un conjunto de datos produce 
una agregación de $x$ con desviación típica $d$. Al realizar 
más ejecuciones del mismo algoritmo se vería \textit{limitado} a tal distribución. Esto es, por 
el Teorema Central del Límite, los 
resultados provenientes de BLB seguirían una distribución normal y entonces, la probabilidad 
de mejorar considerablemente  los resultados de Búsqueda Local 
es la misma que la de encontrar un \textit{outlier} en la distribución 
normal de media $x$ y desviación típica $d$. 




