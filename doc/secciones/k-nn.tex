%%%%%%%%%%%%%%%%%%%%%%%%%%%%%%%%%%%%%%%%%%%%%%%%%%%
%%  Descripción del algoritmo de 1-NN
%%%%%%%%%%%%%%%%%%%%%%%%%%%%%%%%%%%%%%%%%%%%%%%%%%%
\section{ Descripción del algoritmo 1-NN}

Dado un conjunto de datos de entrenamiento etiquetados, se pretende construir un clasificador 
a partir de devolver la categoría más cercana a los datos ya conocidos. 

La implementación del algoritmo es directa dada la descripción, consistiendo en 

\begin{itemize}
    \item Almacenamiento adecuado de los datos de entrenamiento.
    \item Evaluación de la distancia.
\end{itemize}

\subsection{Algoritmo de entrenamiento}

Primero crearemos una estructura que almacena los datos y devuelve
una función dependiente de una función de error 
y el valor que se desea obtener.
% pseudo código cálculo cálculo de gradiente 

\begin{algorithm}
    \caption{Algoritmo de  1-NN}\label{euclid}
    \hspace*{\algorithmicindent} 
        \textbf{Entrada}:
        \begin{itemize}
            \item $x$: Vector a clasificar .
            \item $dist$: Distancia entre vectores $dist$.
            \item $data:$ Matriz  datos , cada columna un atributo distinto, cada fila los atributos del vector $i$-ésimo.
            \item $labels:$ Clase a la que pertenece cada dato (importa la posición).
        \end{itemize}
         datos $data$ y etiquetas $label$.
        
        \hspace*{\algorithmicindent} 
        \textbf{Salida}:
    Devuelve la clase a la clasifica $c$.        
    \begin{algorithmic}[1]
      \Procedure{One-NN}{$d,x$}
          \State $distanciaMínima \gets\infty$
          \State $clase \leftarrow NADA$
        \For{\texttt{all} $(e_i, c_i) \in data \times labels$}
            \If{ $dist(e_i, x) < distanciaMínima$}
                \State $distanciaMínima \gets dist(e_i, x)$
                \State $clase  \gets c_i$
            \EndIf
        \EndFor
        \State \textbf{return} $clase$\Comment{Clase en la que se ha clasificado a $x$}
      \EndProcedure
    \end{algorithmic}
  \end{algorithm}
