%%%%%%%%%%%%%%%%%%%%%%%%%%%%%%%%%%%%%%%%%%%%%%%%%%%
%%  Preprocesado de los datos 
%%%
%%%%%%%%%%%%%%%%%%%%%%%%%%%%%%%%%%%%%%%%%%%%%%%%%%%

\section{Preprocesado de los datos}

Tras leerse de los ficheros y comprobado que no presentan datos perdidos, se han separado los datos y etiquetas y normalizado los distintos atributos de cada dato. 

Además hemos desordenado los datos tras leerlos por si el almacenado de los mismo pudiera guardar algún patrón que afectara perniciosamente al algoritmo.

Los datos a analizar son los siguientes  (obtenidos de la web \url{http://www.ics.uci.edu/~mlearn/MLRepository.html}

\begin{enumerate}
    \item Ionosphere: Conjunto de datos de radar que fueron recogidos por un sistema en Goose Bay, Labrador. 352 ejemplos con 34 características que deben ser clasificados en 2 clases.
    \item Parkinsons: Conjunto de datos orientado a distinguir entre la presencia y la ausencia de la enfermedad de Parkinson en una serie de pacientes a partir de medidas biomédicas de la voz. 195 ejemplos con 22 características que deben ser clasificados en 2 clases.
    \item Spectf-heart: Conjunto de datos de detección de enfermedades cardiacas a partir de imágenes médicas de tomografía computarizada (SPECT) del corazón de pacientes. 267 ejemplos con 44 características que deben ser clasificados en 2 clases.
\end{enumerate}
