%%%%%%%%%%%%%%%%%%%%%%%%%%%%%%%%%%%%%%%%%%%%%%%%%%%
%%  Descripción del algoritmo de 1-NN
%%%%%%%%%%%%%%%%%%%%%%%%%%%%%%%%%%%%%%%%%%%%%%%%%%%
\section{ Descripción del algoritmo 1-NN}

Dado un conjunto de datos de entrenamiento etiquetados, se pretende construir un clasificador 
a partir de devolver la categoría más cercana a los datos ya conocidos. 

La implementación del algoritmo es directa dada la descripción, consistiendo en 

\begin{itemize}
    \item Almacenamiento adecuado de los datos de entrenamiento.
    \item Evaluación de la distancia.
\end{itemize}

\subsection{Algoritmo 1-NN}

Primero crearemos una estructura que almacena los datos y devuelve
una función dependiente de una función de error 
y el valor que se desea obtener.
% pseudo código cálculo cálculo de gradiente 

\begin{algorithm}
    \caption{Algoritmo de  1-NN}\label{euclid}
    \hspace*{\algorithmicindent} 
        \textbf{Entrada}:
        \begin{itemize}
            \item $x$: Vector a clasificar .
            \item $dist$: Distancia entre vectores $dist$.
            \item $data:$ Matriz  datos , cada columna un atributo distinto, cada fila los atributos del vector $i$-ésimo.
            \item $labels:$ Vector de clases a la que pertenece  cada dato (elemento i-ésimo corresponde al dato iésimo).
        \end{itemize}
        \hspace*{\algorithmicindent} 
        \textbf{Salida}:
    Devuelve la clase a la que clasifica $x$.        
    \begin{algorithmic}[1]
      \Procedure{OneNN}{$x, dist, data,labels$}
          \State $distanciaMínima \gets\infty$
          \State $clase \leftarrow NADA$
        \For{\texttt{all} $(e_i, c_i) \in data \times labels$} \Comment{Recorremos cada vector de atributo con su clase asociada}
            \If{ $dist(e_i, x) < distanciaMínima$}
                \State $distanciaMínima \gets dist(e_i, x)$
                \State $clase  \gets c_i$
            \EndIf
        \EndFor
        \State \textbf{return} $clase$\Comment{Clase en la que se ha clasificado a $x$}
      \EndProcedure
    \end{algorithmic}
  \end{algorithm}  

  Notemos que la complejidad del algoritmo es de orden $\mathcal{O}(n)$ con $n$ el tamaño 
  del conjunto de entrenamiento.

\subsection{Resultados para el 1-NN sin pesos} 

  Para la distancia euclidea sin ninguna ponderación de atributos, es decir, con 
  $w = (1,\ldots, 1)$ se han obtenido los siguientes resultados: 
  
  \begin{table}[h]
    \centering
    \resizebox{\columnwidth}{!}{%
      \begin{tabular}{|c|c|c|c|c|c|c|}
     \hline
     & \multicolumn{2}{|c|}{\textit{Ionosphere}} 
     & \multicolumn{2}{|c|}{\textit{Parkinsons}} 
     &\multicolumn{2}{|c|}{\textit{Spectf-heart}}\\
     \hline
       & Clasificación  & Tiempo (ms)
       & Clasificación	& Tiempo (ms)
       & Clasificación	& Tiempo (ms)
       \\ \hline
Partición 1 
    & 84.507  & 2.258
    &	94.872 &	0.529
    & 84.286	& 2.408
\\
Partición 2 
    & 87.143   & 2.238
    & 100.0	 & 9.314
    & 78.571	& 9.923

\\
Partición 3 
    & 80.0  & 14.370
    & 94.872 &	0.592 
    & 81.429 &	2.729

\\
Partición 4 
  & 88.571  & 1.929
  & 92.308 &	0.615 
  & 85.714 &	2.695

\\
Partición 5 
  & 90.0  & 2.323
  & 97.4359 &	0.513
  & 82.609 &	2.787

\\ \hline
Medias 
  & 86.0442  & 4.623
  &	95.897 & 2.312
  & 82.522 & 4.109

\\ \hline
Desviación típica 
  & 3.941  & 3.743
  &	2.923 & 0.023
  & 2.745 & 3.637
\\
  \hline        
      \end{tabular}
  }
      \caption{Métricas de clasificación y tiempo  para el algoritmo 1-NN sin ponderación de pesos en los diferentes conjuntos de datos}
      \label{Tabla:1nn-base}
\end{table}

Utilizaremos los resultado de la tabla \ref{Tabla:1nn-base} como base para comparar la bondad de otras técnicas.

\subsubsection*{Reflexiones}  

A la vista de los resultados en clasificación en las tres bases de datos, con más de un $80\%$ de
 precisión y desviaciones típicas de menos de $4\%$, 
podría uno dejarse llevar por la particularidad y pensar que el algoritmo $1-$NN es bastante robusto 
y produce buenos resultados. Sin embargo estos no dejan de ser casos concretos y existirán conjuntos para el que falle, de hecho es fácil pensar un ejemplo: 

Consideramos $esPar:\mathbb{N} \longrightarrow \{0,1\}$
la regla subyacente que clasifica a nuestros datos y consideramos el conjunto de datos 
los conjuntos de entrenamiento son los ciclos generados por 
$<2>+2 = \{2^1, 2^2, 2^3, \ldots\}$ cuya etiqueta es $1$ y por otra parte
$<11>+11 = \{11, 11^2, 11^3, \ldots\}$ cuya etiqueta claramente es $0$.
Esta claro que existe el mismo número de números pares que impares, pero ante estos conjuntos es más probable que se clasifique como par.



