\documentclass[11pt,twoside,titlepage,a4paper]{article}

%%%%%%%%%%%%%%%%%%%%%%%%%%%%%%%%%%%%%%%%%%%%%%%%%%%%%%%%%%%%%%%%%%%%%%%%%%%%%%%
% PAQUETES
%salu2 - Valentino
\usepackage{xcolor} % Colores
\usepackage[xcolor]{mdframed} % Marcos
\usepackage{amsmath} % Matemáticas
\usepackage{amsfonts} % Letras caligráficas para matemáticas
\usepackage{mathtools} % Matemáticas extra
\usepackage{amsthm} % Teoremas
\usepackage{listingsutf8} % Código
\usepackage[a4paper]{geometry} % Márgenes
\usepackage{enumitem} % Opciones de personalización de listas
\usepackage{fancyhdr} % Encabezado / Pie de página
\usepackage{titlesec} % Títulos
\usepackage{pagecolor} % Colorear las portadas
\usepackage{graphicx} % Imágenes
\usepackage{hyperref} % Referencias
\usepackage{sidenotes} % Notas en el margen
\usepackage{pgfplots} % Gráficos de funciones
\usepackage{biblatex} % Bibliografía
\bibliography{bibliografia.bib}
\usepackage{caption}
\usepackage{subcaption}
% Para escribir pseudocódigo
%\usepackage{algorithm}  
%\usepackage{algorithmic} 
\usepackage{algorithm}% http://ctan.org/pkg/algorithm
\usepackage{algpseudocode}% http://ctan.org/pkg/algorithmicx
%%%%%%%%%%%%%%%%%%%%%%%%%%%%%%%%%%%%%%%%%%%%%%%%%%%%%%%%%%%%%%%%%%%%%%%%%%%%%%%
% COMANDOS PERSONALIZADOS

% Año o cualquier otra información para la portada
\newcommand{\fecha}{
\today
}
% Autores del documento
\newcommand{\autores}{
 Blanca Cano Camarero
}
% DNI
\newcommand{\DNI}{
 75577392J
}
% Email 
\newcommand{\email}{
 blancacano@correo.ugr.es
}
% Grupo de prácticas 
\newcommand{\grupo}{
 Grupo de prácticas 2, lunes.
}

\newcommand{\margenimagen}{
\newgeometry{
    left=2.5cm, % Margen izquierdo
	right=5cm, % Margen derecho
	bottom=2.5cm % Margen inferior}
}
}

%%%%%%%%%%%%%%%%%%%%%%%%%%%%%%%%%%%%%%%%%%%%%%%%%%%%%%%%%%%%%%%%%%%%%%%%%%%%%%%
% TIPOGRAFÍA

\usepackage{heuristica}
\usepackage[heuristica,vvarbb,bigdelims]{newtxmath}
\usepackage[T1]{fontenc}
\renewcommand*\oldstylenums[1]{\textosf{#1}}
\usepackage[spanish]{babel}

%%%%%%%%%%%%%%%%%%%%%%%%%%%%%%%%%%%%%%%%%%%%%%%%%%%%%%%%%%%%%%%%%%%%%%%%%%%%%%%
% DEFINICIÓN DE COLORES

% COLORES DE LA ESTRUCTURA DEL DOCUMENTO
%\definecolor{bg_por}{HTML}{8A0808} % Portada
\definecolor{bg_por}{HTML}{95b2b0} % Portada
\definecolor{fg_por}{HTML}{FFFFFF} % Texto de la portada
\definecolor{fg_sec}{HTML}{8A0808} % Títulos de las secciones
\definecolor{fg_ssec}{HTML}{610505} % Títulos de las subsecciones
\definecolor{fg_sssec}{HTML}{300303} % Títulos de las subsubsecciones
\definecolor{fg_head}{HTML}{610B0B} % Texto del encabezado
% COLORES PARA CÓDIGO
\definecolor{li_code}{HTML}{8A0808} % Línea a la izquierda
\definecolor{rw_code}{HTML}{610B0B} % Palabras reservadas
\definecolor{st_code}{HTML}{300303} % Cadenas de caracteres
\definecolor{cm_code}{HTML}{333333} % Comentarios
% COLORES PARA LISTAS
\definecolor{l_1}{HTML}{8A0808} % Primer símbolo
\definecolor{l_2}{HTML}{610505} % Primera indentación
\definecolor{l_3}{HTML}{300303} % Segunda indentación
\definecolor{l_4}{HTML}{000000} % Tercera indentación
% COLORES PARA MARCOS
\definecolor{li_ejs}{HTML}{8A0808} % Línea marco ejemplos
\definecolor{li_defs}{HTML}{610505} % Línea marco definiciones
\definecolor{bg_ejs}{HTML}{FFEDEE} % Fondo ejemplos
\definecolor{bg_defs}{HTML}{FFE0DF} % Fondo definiciones
% COLORES PARA REFERENCIAS
\definecolor{fg_url}{HTML}{610505} % Links


%%%%%%%%%%%%%%%%%%%%%%%%%%%%%%%%%%%%%%%%%%%%%%%%%%%%%%%%%%%%%%%%%%%%%%%%%%%%%%%
% REFERENCIAS

\hypersetup{
	pdftitle={	Implementación de las técnicas de mezcla de regiones usando la ecuación de Poisson}, % Título del pdf
	pdfauthor={}, % Autor del pdf
    colorlinks=true, % Referencias con color
    linkcolor=black, % Color de las referencias internas
    urlcolor=fg_url, % Color de los links
    citecolor=fg_ssec, % Color de las referencias
}
\urlstyle{same} % Links con el mismo tipo de letra

%%%%%%%%%%%%%%%%%%%%%%%%%%%%%%%%%%%%%%%%%%%%%%%%%%%%%%%%%%%%%%%%%%%%%%%%%%%%%%%
% TEOREMAS

\numberwithin{equation}{section} % Numeración de ecuaciones
% Teoremas-Lemas-Definiciones-Corolarios
\newtheoremstyle{usual} % Nombre del estilo
{} % Espacio por encima
{} % Espacio por debajo
{} % Estilo del cuerpo
{} % Indentación
{\bfseries} % Estilo de la cabecera
{} % Símbolo tras la cabecera
{ } % Espacio tras la cabecera
{\thmname{#1}\thmnumber{ #2 }\thmnote{(\textit{#3})}:} % Especificación de la cabecera
\theoremstyle{usual}
\newtheorem{theorem}{Teorema}[section] % Comando para los teoremas

%%%%%%%%%%%%%%%%%%%%%%%%%%%%%%%%%%%%%%%%%%%%%%%%%%%%%%%%%%%%%%%%%%%%%%%%%%%%%%%
% CÓDIGO

\lstset{
	basicstyle=\footnotesize\ttfamily, % Estilo del código
	inputencoding=utf8/latin1, % Codificación
	xleftmargin=1.3em, % Margen extra a la izquierda
	breaklines=true, % Romper líneas largas
	language=, % Lenguaje del código
	numbers=left, % Números de línea
	numbersep=8pt, % Separación de los números de línea
	tabsize=4, % Tamaño de los tabs
	frame=leftline, % Posición del enmarcado
	framerule=1pt, % Grosor del enmarcado
	showstringspaces=false, % Mostrar los espacios en las cadenas de caracteres
	keywordstyle=\color{rw_code}, % Estilo de las palabras reservadas
	numberstyle=\normalfont, % Estilo de los números de línea
	rulecolor=\color{li_code}, % Estilo del enmarcado
	commentstyle=\color{cm_code}, % Estilo de los comentarios
	stringstyle=\color{st_code} % Estilo de las cadenas de caracteres
}

%%%%%%%%%%%%%%%%%%%%%%%%%%%%%%%%%%%%%%%%%%%%%%%%%%%%%%%%%%%%%%%%%%%%%%%%%%%%%%%
% MÁRGENES

\geometry{
	left=2.5cm, % Margen izquierdo
	right=2.5cm, % Margen derecho
	bottom=2.5cm % Margen inferior
}

%%%%%%%%%%%%%%%%%%%%%%%%%%%%%%%%%%%%%%%%%%%%%%%%%%%%%%%%%%%%%%%%%%%%%%%%%%%%%%%
% LISTAS/TABLAS

\renewcommand{\arraystretch}{1.3} % Tamaño entre líneas de una tabla
% SÍMBOLOS LISTAS
\renewcommand{\labelitemi}{\color{l_1}$\bullet$} % Primer símbolo
\renewcommand{\labelitemii}{\color{l_2}$\circ$} % Símbolo primera indentación
\renewcommand{\labelitemiii}{\color{l_3}$\diamond$} % Símbolo segunda indentación
\renewcommand{\labelitemiv}{\color{l_4}$-$} % Símbolo tercera indentación
% SÍMBOLOS ENUMERACIONES
\renewcommand{\labelenumi}{\color{l_1}\bfseries\arabic{enumi}.} % Primer símbolo
\renewcommand{\labelenumii}{\color{l_2}\bfseries\Roman{enumii}.} % Símbolo primera indentación
\renewcommand{\labelenumiii}{\color{l_3}\bfseries(\alph{enumiii})} % Símbolo segunda indentación
\renewcommand{\labelenumiv}{\color{l_4}\bfseries\Alph{enumiv}.} % Símbolo tercera indentación
% DESCRIPCIONES
\renewcommand{\descriptionlabel}[1]{\hspace{\labelsep}\color{l_1}\textbf{#1}} % Color y estilo del título de la descripción

%%%%%%%%%%%%%%%%%%%%%%%%%%%%%%%%%%%%%%%%%%%%%%%%%%%%%%%%%%%%%%%%%%%%%%%%%%%%%%%
% ENCABEZADO/PIE DE PAGINA

\setlength{\headheight}{14pt} % Tamaño del encabezado
\pagestyle{fancy}
\fancyhf{}
% Para que aparezca el título de la sección y no el número 
\renewcommand{\sectionmark}[1]{%
\markboth{#1}{}}
% Encabezado
\fancyhead[LE,RO]{\color{fg_head}{\leftmark}} % A la izquierda en pares, derecha en impares
\fancyhead[RE,LO]{\color{fg_head}{}} % A la derecha en pares, izquierda en impares
% Pie de página
\fancyfoot[LE,RO]{\Large\textbf{\thepage}} % A la izquierda en pares, derecha en impares
\renewcommand{\headrulewidth}{0.5pt} % Grosor de la línea

%%%%%%%%%%%%%%%%%%%%%%%%%%%%%%%%%%%%%%%%%%%%%%%%%%%%%%%%%%%%%%%%%%%%%%%%%%%%%%%
% TÍTULOS

% Estilo de las secciones
\titleformat{\section}
{\color{fg_sec}\Huge\bfseries}
{\color{fg_sec}\thesection}{1em}{}
% Estilo de las subsecciones
\titleformat{\subsection}
{\color{fg_ssec}\huge\bfseries}
{\color{fg_ssec}\thesubsection}{1em}{}
% Estilo de las subsecciones
\titleformat{\subsubsection}
{\color{fg_sssec}\LARGE\bfseries}
{\color{fg_sssec}\thesubsubsection}{1em}{}
%%%%%%%%%%%%%%%%%%%%%%%%%%%%%%%%%%%%%%%%%%%%%%%%%%%%%%%%%%%%%%%%%%%%%%%%%%%%%%%
% Atajos


\newcommand{\R}{\mathbb{R}}
%%%%%%%%%%%%%%%%%%%%%%%%%%%%%%%%%%%%%%%%%%%%%%%%%%%%%%%%%%%%%%%%%%%%%%%%%%%%%%%
% MISCELÁNEO

\renewcommand{\contentsname}{Índice} % Cambiar el título del índice
\setlength\parindent{0pt} % Tamaño de la sangría


\begin{document}
%%%%%%%%%%%%%%%%%%%%%%%%%%%%%%%%%%%%%%%%%%%%%%%%%%%%%%%%%%%%%%%%%%%%%%%%%%%%%%%
% PORTADA

\begin{titlepage}
	\newpagecolor{bg_por} % Color de la portada
	\centering
	%\includegraphics[width=0.7\textwidth]{imagenes/ugr_logo.jpg} \\ % Logo
	\vspace{7em}
	\centering
	\color{fg_por}{
	%\title{Implementación de la técnica de mezclado de colores usando la ecuación del Poisson}
		\fontsize{50pt}{50pt}{\scshape{
		Técnica de Búsqueda Local\\ 
        y \\
        Algoritmos Greedy\\
        para el Problema del Aprendizaje \\
        de Pesos 
        en Características.}} % Título
	}
	\vfill
	\centering
	\color{fg_por}{\large{\autores}} \\
	\vspace{2em}
	\color{fg_por}{\large{DNI: \DNI}} \\
	\vspace{2em}
	\color{fg_por}{\large{correo: \email}} \\
	\vspace{2em}
	\color{fg_por}{\large{\grupo}} \\
	\vspace{2em}
	\color{fg_por}{\Large{\textit{\fecha}}} \\
\end{titlepage}
\restorepagecolor

%%%%%%%%%%%%%%%%%%%%%%%%%%%%%%%%%%%%%%%%%%%%%%%%%%%%%%%%%%%%%%%%%%%%%%%%%%%%%%%
% ÍNDICE

\newpage
\tableofcontents
\newpage
\listoffigures
\clearpage

%%%%%%%%%%%%%%%%%%%%%%%%%%%%%%

%% Añadir aquí las respectivas entradas 
%%%%%%%%%%%%%%%%%%%%%%%%%%%%%%%%%%%%%%%%%%%%%%%%%%%%%%%%%%%%%
%%% DESCRIPCIÓN DEL PROBLEMA 
%%%%%%%%%%%%%%%%%%%%%%%%%%%%%%%%%%%%%%%%%%%%%%%%%%%%%%%%%%%%%

\section{Descripción del problema}  

El problema de Aprendizaje de Pesos en Características (APC) es un problema de aprendizaje en clasificación para el cual se disponen de una muestra de objetos $\{w_i\}_{1 \leq i \leq N}$ 
representados en función de sus valores en una serie de atributos dados por $x$:
\begin{equation*}
    w_i \text{ tiene asociado el vector de $t$ atributos } (x_1(w_i), \ldots, x_t(w_i)).
\end{equation*}
Cada objeto pertenece a una de las $m$ clases existentes 
$\{C_1, \ldots, C_m\}$.

El objetivo de nuestro problema es poder clasificar cualquier $w$ de un modo automático.

Trataremos de afrontar el problema mediante el algoritmo de clasificación 1-NN, es decir clasificar cierto w a partir de la etiqueta del vector más cercano a él que se conozca.
%%%%%%%%%%%%%%%%%%%%%%%%%%%%%%%%%%%%%%%%%%%%%%%%%%%
%%  Preprocesado de los datos 
%%%
%%%%%%%%%%%%%%%%%%%%%%%%%%%%%%%%%%%%%%%%%%%%%%%%%%%

\section{Preprocesado de los datos}

Tras leerse de los ficheros y comprobado que no presentan datos perdidos, se han separado los datos y etiquetas y normalizado los distintos atributos de cada dato. 

Hay un caso en que los valores son tan parecidos que la diferencia entre el mínimo y el máximo son iguales como criterio se ha dejado los datos a cero.


%%%%%%%%%%%%%%%%%%%%%%%%%%%%%%%%%%%%%%%%%%%%%%%%%%%
%%  Descripción del algoritmo de 1-NN
%%%%%%%%%%%%%%%%%%%%%%%%%%%%%%%%%%%%%%%%%%%%%%%%%%%
\section{ Descripción del algoritmo 1-NN}

Dado un conjunto de datos de entrenamiento etiquetados, se pretende construir un clasificador 
a partir de devolver la categoría más cercana a los datos ya conocidos. 

La implementación del algoritmo es directa dada la descripción, consistiendo en 

\begin{itemize}
    \item Almacenamiento adecuado de los datos de entrenamiento.
    \item Evaluación de la distancia.
\end{itemize}

\subsection{Algoritmo 1-NN}

Primero crearemos una estructura que almacena los datos y devuelve
una función dependiente de una función de error 
y el valor que se desea obtener.
% pseudo código cálculo cálculo de gradiente 

\begin{algorithm}
    \caption{Algoritmo de  1-NN}\label{euclid}
    \hspace*{\algorithmicindent} 
        \textbf{Entrada}:
        \begin{itemize}
            \item $x$: Vector a clasificar .
            \item $dist$: Distancia entre vectores $dist$.
            \item $data:$ Matriz  datos , cada columna un atributo distinto, cada fila los atributos del vector $i$-ésimo.
            \item $labels:$ Vector de clases a la que pertenece  cada dato (elemento i-ésimo corresponde al dato iésimo).
        \end{itemize}
        \hspace*{\algorithmicindent} 
        \textbf{Salida}:
    Devuelve la clase a la que clasifica $x$.        
    \begin{algorithmic}[1]
      \Procedure{OneNN}{$x, dist, data,labels$}
          \State $distanciaMínima \gets\infty$
          \State $clase \leftarrow NADA$
        \For{\texttt{all} $(e_i, c_i) \in data \times labels$} \Comment{Recorremos cada vector de atributo con su clase asociada}
            \If{ $dist(e_i, x) < distanciaMínima$}
                \State $distanciaMínima \gets dist(e_i, x)$
                \State $clase  \gets c_i$
            \EndIf
        \EndFor
        \State \textbf{return} $clase$\Comment{Clase en la que se ha clasificado a $x$}
      \EndProcedure
    \end{algorithmic}
  \end{algorithm}  

\subsection{Resultados para el 1-NN sin pesos} 

  Para la distancia euclidea sin ninguna ponderación de atributos, es decir, con 
  $w = (1,\ldots, 1)$ se han obtenido los siguientes resultados: 
  
 % \resizebox{\textwidth}{!}{

  \begin{table}[h]
    \centering
    \resizebox{\columnwidth}{!}{%
      \begin{tabular}{|c|c|c|c|c|c|c|}
     \hline
     & \multicolumn{2}{|c|}{\textit{Ionosphere}} 
     & \multicolumn{2}{|c|}{\textit{Parkinsons}} 
     &\multicolumn{2}{|c|}{\textit{Spectf-heart}}\\
     \hline
       & Clasificación  & Tiempo (ms)
       & Clasificación	& Tiempo (ms)
       & Clasificación	& Tiempo (ms)
       \\ \hline
Partición 1 
    & 84.507  & 2.258
    &	94.872 &	0.529
    & 84.286	& 2.408
\\
Partición 2 
    & 87.143   & 2.238
    & 100.0	 & 9.314
    & 78.571	& 9.923

\\
Partición 3 
    & 80.0  & 14.370
    & 94.872 &	0.592 
    & 81.429 &	2.729

\\
Partición 4 
  & 88.571  & 1.929
  & 92.308 &	0.615 
  & 85.714 &	2.695

\\
Partición 5 
  & 90.0  & 2.323
  & 97.4359 &	0.513
  & 82.609 &	2.787

\\ \hline
Medias 
  & 86.0442  & 4.623
  &	95.897 & 2.312
  & 82.522 & 4.109

\\ \hline
Desviación típica 
  & 3.941  & 3.743
  &	2.923 & 0.023
  & 2.745 & 3.637
\\
  \hline        
      \end{tabular}
  }
      \caption{Métricas de clasificación y tiempo  para el algoritmo 1-NN sin ponderación de pesos en los diferentes conjuntos de datos}
      \label{Tabla:1nn-base}
\end{table}

Utilizaremos los resultado de la tabla \ref{Tabla:1nn-base} como base para comparar la bondad de otras técnicas.

\subsubsection*{Reflexiones}  

A la vista de los resultados en clasificación en las tres bases de datos, con más de un $80\%$ de
 precisión y desviaciones típicas de menos de $4\%$, 
podría uno dejarse llevar por la particularidad y pensar que el algoritmo $1-$NN es bastante robusto 
y produce buenos resultados. Sin embargo estos no dejan de ser casos concretos y existirán conjuntos para el que falle, de hecho es fácil pensar un ejemplo: 

Consideramos $esPar:\mathbb{N} \longrightarrow \{0,1\}$
la regla subyacente que clasifica a nuestros datos y consideramos el conjunto de datos 
los conjuntos de entrenamiento son los ciclos generados por 
$<2>+2 = \{2^1, 2^2, 2^3, \ldots\}$ cuya etiqueta es $1$ y por otra parte
$<11>+11 = \{11, 11^2, 11^3, \ldots\}$ cuya etiqueta claramente es $0$.
Esta claro que existe el mismo número de números pares que impares, pero ante estos conjuntos es más probable que se clasifique como par.




\section{Algoritmo de búsqueda local}  

Si como distancia en el problema anterior tomamos 
\begin{equation}
    d_e(e_1, e_2)
     = 
     \sqrt{
         \sum_i
             w_i \cdot (e^i_1 - e^i_2)^2  
        +
        \sum_j w_j \cdot d_h(e_1^j, e_2^j)
         }
\end{equation}

El vector de pesos $w \in [0,1]^n$ es nuestra incógnita
y deberemos de encontrar el que mejore el clasificador de ponderación 1.

\subsection{Componentes del algoritmo de búsqueda local}  

Las métricas a observar de nuestro algoritmo son: 

\begin{itemize}
    \item \textbf{Precisión} o \texttt{tasa-clasificación} Rendimiento promedio para $k=1$ y utilizado \textit{leave one out}.
    \item \textbf{Tasa de reducción}, número de características que se consideran como clasificador. 
    \begin{equation}
        \text{tasa-reducción} 
        = 
        100
        \frac{\text{número de }w_i < 0.1}{\text{número de características}}.
    \end{equation}
    \item \textbf{Función de evaluación} Permite cuantificar el éxito de nuestra selección de pesos de acorde a una  combinación de precisión y simplicidad, su expresión viene dada por 
    \begin{equation}
        F(w) = 
            \alpha \texttt{tasa-clasificación}
            +
            (1 - \alpha) \texttt{tasa-reducción}.
    \end{equation}
    Nótese que $\alpha$ es la ponderación de relevancia que se le da al modelo.
\end{itemize}

\subsection{Búsqueda local del primero mejor}  

El algoritmo que vamos a usar es el conocido como el del \textit{primero mejor} y que radica en esencia de que  cuanto se genera 
una solución que mejora a la actual se aplica el movimiento y se pasa a la siguiente iteración.

Se detiene la búsqueda cuando se haya generado un número máximo de vecinos que no mejora el resultado.

Descripción del algoritmo, necesitamos primero tener una función para general vecinos, de acorde a los requisitos, esta será 
la función \texttt{GeneraVecino($w,\sigma$)} que devolverá un vector 

\textcolor{red}{El vector no puede tomar valores negativos} y además serán menor que 1.
\begin{algorithm}
    \caption{Búsqueda local del primero mejor}
    \hspace*{\algorithmicindent} 
        \textbf{Entrada}:
        \hspace*{\algorithmicindent} 
        \textbf{Salida}:
        Vector de pesos $w$.        
    \begin{algorithmic}[1]
      \Procedure{PrimeroMejor}{numeroMáximoVecinos, evaluacionesMáximas}
            \Comment{Inicializamos pesos}
          \State $w \gets$ vector aleatorio  
          \State $iteraciones \gets 0$ \Comment{Indica número de generaciones}
        \While{iteraciones < iteracionesMáximas}

        \EndWhile
        \State \textbf{return} $clase$\Comment{Clase en la que se ha clasificado a $x$}
      \EndProcedure
    \end{algorithmic}
  \end{algorithm}






\end{document}