\documentclass[11pt,twoside,titlepage,a4paper]{article}

%%%%%%%%%%%%%%%%%%%%%%%%%%%%%%%%%%%%%%%%%%%%%%%%%%%%%%%%%%%%%%%%%%%%%%%%%%%%%%%
% PAQUETES
%salu2 - Valentino
\usepackage{xcolor} % Colores
\usepackage[xcolor]{mdframed} % Marcos
\usepackage{amsmath} % Matemáticas
\usepackage{amsfonts} % Letras caligráficas para matemáticas
\usepackage{mathtools} % Matemáticas extra
\usepackage{amsthm} % Teoremas
\usepackage{listingsutf8} % Código
\usepackage[a4paper]{geometry} % Márgenes
\usepackage{enumitem} % Opciones de personalización de listas
\usepackage{fancyhdr} % Encabezado / Pie de página
\usepackage{titlesec} % Títulos
\usepackage{pagecolor} % Colorear las portadas
\usepackage{graphicx} % Imágenes
\usepackage{hyperref} % Referencias
\usepackage{sidenotes} % Notas en el margen
\usepackage{pgfplots} % Gráficos de funciones
\usepackage{biblatex} % Bibliografía
\bibliography{bibliografia.bib}
\usepackage{caption}
\usepackage{subcaption}
% Para escribir pseudocódigo
%\usepackage{algorithm}  
%\usepackage{algorithmic} 
\usepackage{algorithm}% http://ctan.org/pkg/algorithm
\usepackage{algpseudocode}% http://ctan.org/pkg/algorithmicx
%%%%%%%%%%%%%%%%%%%%%%%%%%%%%%%%%%%%%%%%%%%%%%%%%%%%%%%%%%%%%%%%%%%%%%%%%%%%%%%
% COMANDOS PERSONALIZADOS

% Año o cualquier otra información para la portada
\newcommand{\fecha}{
\today
}
% Autores del documento
\newcommand{\autores}{
 Blanca Cano Camarero
}

\newcommand{\margenimagen}{
\newgeometry{
    left=2.5cm, % Margen izquierdo
	right=5cm, % Margen derecho
	bottom=2.5cm % Margen inferior}
}
}

%%%%%%%%%%%%%%%%%%%%%%%%%%%%%%%%%%%%%%%%%%%%%%%%%%%%%%%%%%%%%%%%%%%%%%%%%%%%%%%
% TIPOGRAFÍA

\usepackage{heuristica}
\usepackage[heuristica,vvarbb,bigdelims]{newtxmath}
\usepackage[T1]{fontenc}
\renewcommand*\oldstylenums[1]{\textosf{#1}}
\usepackage[spanish]{babel}

%%%%%%%%%%%%%%%%%%%%%%%%%%%%%%%%%%%%%%%%%%%%%%%%%%%%%%%%%%%%%%%%%%%%%%%%%%%%%%%
% DEFINICIÓN DE COLORES

% COLORES DE LA ESTRUCTURA DEL DOCUMENTO
%\definecolor{bg_por}{HTML}{8A0808} % Portada
\definecolor{bg_por}{HTML}{95b2b0} % Portada
\definecolor{fg_por}{HTML}{FFFFFF} % Texto de la portada
\definecolor{fg_sec}{HTML}{8A0808} % Títulos de las secciones
\definecolor{fg_ssec}{HTML}{610505} % Títulos de las subsecciones
\definecolor{fg_sssec}{HTML}{300303} % Títulos de las subsubsecciones
\definecolor{fg_head}{HTML}{610B0B} % Texto del encabezado
% COLORES PARA CÓDIGO
\definecolor{li_code}{HTML}{8A0808} % Línea a la izquierda
\definecolor{rw_code}{HTML}{610B0B} % Palabras reservadas
\definecolor{st_code}{HTML}{300303} % Cadenas de caracteres
\definecolor{cm_code}{HTML}{333333} % Comentarios
% COLORES PARA LISTAS
\definecolor{l_1}{HTML}{8A0808} % Primer símbolo
\definecolor{l_2}{HTML}{610505} % Primera indentación
\definecolor{l_3}{HTML}{300303} % Segunda indentación
\definecolor{l_4}{HTML}{000000} % Tercera indentación
% COLORES PARA MARCOS
\definecolor{li_ejs}{HTML}{8A0808} % Línea marco ejemplos
\definecolor{li_defs}{HTML}{610505} % Línea marco definiciones
\definecolor{bg_ejs}{HTML}{FFEDEE} % Fondo ejemplos
\definecolor{bg_defs}{HTML}{FFE0DF} % Fondo definiciones
% COLORES PARA REFERENCIAS
\definecolor{fg_url}{HTML}{610505} % Links


%%%%%%%%%%%%%%%%%%%%%%%%%%%%%%%%%%%%%%%%%%%%%%%%%%%%%%%%%%%%%%%%%%%%%%%%%%%%%%%
% REFERENCIAS

\hypersetup{
	pdftitle={	Implementación de las técnicas de mezcla de regiones usando la ecuación de Poisson}, % Título del pdf
	pdfauthor={}, % Autor del pdf
    colorlinks=true, % Referencias con color
    linkcolor=black, % Color de las referencias internas
    urlcolor=fg_url, % Color de los links
    citecolor=fg_ssec, % Color de las referencias
}
\urlstyle{same} % Links con el mismo tipo de letra

%%%%%%%%%%%%%%%%%%%%%%%%%%%%%%%%%%%%%%%%%%%%%%%%%%%%%%%%%%%%%%%%%%%%%%%%%%%%%%%
% TEOREMAS

\numberwithin{equation}{section} % Numeración de ecuaciones
% Teoremas-Lemas-Definiciones-Corolarios
\newtheoremstyle{usual} % Nombre del estilo
{} % Espacio por encima
{} % Espacio por debajo
{} % Estilo del cuerpo
{} % Indentación
{\bfseries} % Estilo de la cabecera
{} % Símbolo tras la cabecera
{ } % Espacio tras la cabecera
{\thmname{#1}\thmnumber{ #2 }\thmnote{(\textit{#3})}:} % Especificación de la cabecera
\theoremstyle{usual}
\newtheorem{theorem}{Teorema}[section] % Comando para los teoremas

%%%%%%%%%%%%%%%%%%%%%%%%%%%%%%%%%%%%%%%%%%%%%%%%%%%%%%%%%%%%%%%%%%%%%%%%%%%%%%%
% CÓDIGO

\lstset{
	basicstyle=\footnotesize\ttfamily, % Estilo del código
	inputencoding=utf8/latin1, % Codificación
	xleftmargin=1.3em, % Margen extra a la izquierda
	breaklines=true, % Romper líneas largas
	language=, % Lenguaje del código
	numbers=left, % Números de línea
	numbersep=8pt, % Separación de los números de línea
	tabsize=4, % Tamaño de los tabs
	frame=leftline, % Posición del enmarcado
	framerule=1pt, % Grosor del enmarcado
	showstringspaces=false, % Mostrar los espacios en las cadenas de caracteres
	keywordstyle=\color{rw_code}, % Estilo de las palabras reservadas
	numberstyle=\normalfont, % Estilo de los números de línea
	rulecolor=\color{li_code}, % Estilo del enmarcado
	commentstyle=\color{cm_code}, % Estilo de los comentarios
	stringstyle=\color{st_code} % Estilo de las cadenas de caracteres
}

%%%%%%%%%%%%%%%%%%%%%%%%%%%%%%%%%%%%%%%%%%%%%%%%%%%%%%%%%%%%%%%%%%%%%%%%%%%%%%%
% MÁRGENES

\geometry{
	left=2.5cm, % Margen izquierdo
	right=2.5cm, % Margen derecho
	bottom=2.5cm % Margen inferior
}

%%%%%%%%%%%%%%%%%%%%%%%%%%%%%%%%%%%%%%%%%%%%%%%%%%%%%%%%%%%%%%%%%%%%%%%%%%%%%%%
% LISTAS/TABLAS

\renewcommand{\arraystretch}{1.3} % Tamaño entre líneas de una tabla
% SÍMBOLOS LISTAS
\renewcommand{\labelitemi}{\color{l_1}$\bullet$} % Primer símbolo
\renewcommand{\labelitemii}{\color{l_2}$\circ$} % Símbolo primera indentación
\renewcommand{\labelitemiii}{\color{l_3}$\diamond$} % Símbolo segunda indentación
\renewcommand{\labelitemiv}{\color{l_4}$-$} % Símbolo tercera indentación
% SÍMBOLOS ENUMERACIONES
\renewcommand{\labelenumi}{\color{l_1}\bfseries\arabic{enumi}.} % Primer símbolo
\renewcommand{\labelenumii}{\color{l_2}\bfseries\Roman{enumii}.} % Símbolo primera indentación
\renewcommand{\labelenumiii}{\color{l_3}\bfseries(\alph{enumiii})} % Símbolo segunda indentación
\renewcommand{\labelenumiv}{\color{l_4}\bfseries\Alph{enumiv}.} % Símbolo tercera indentación
% DESCRIPCIONES
\renewcommand{\descriptionlabel}[1]{\hspace{\labelsep}\color{l_1}\textbf{#1}} % Color y estilo del título de la descripción

%%%%%%%%%%%%%%%%%%%%%%%%%%%%%%%%%%%%%%%%%%%%%%%%%%%%%%%%%%%%%%%%%%%%%%%%%%%%%%%
% ENCABEZADO/PIE DE PAGINA

\setlength{\headheight}{14pt} % Tamaño del encabezado
\pagestyle{fancy}
\fancyhf{}
% Para que aparezca el título de la sección y no el número 
\renewcommand{\sectionmark}[1]{%
\markboth{#1}{}}
% Encabezado
\fancyhead[LE,RO]{\color{fg_head}{\leftmark}} % A la izquierda en pares, derecha en impares
\fancyhead[RE,LO]{\color{fg_head}{}} % A la derecha en pares, izquierda en impares
% Pie de página
\fancyfoot[LE,RO]{\Large\textbf{\thepage}} % A la izquierda en pares, derecha en impares
\renewcommand{\headrulewidth}{0.5pt} % Grosor de la línea

%%%%%%%%%%%%%%%%%%%%%%%%%%%%%%%%%%%%%%%%%%%%%%%%%%%%%%%%%%%%%%%%%%%%%%%%%%%%%%%
% TÍTULOS

% Estilo de las secciones
\titleformat{\section}
{\color{fg_sec}\Huge\bfseries}
{\color{fg_sec}\thesection}{1em}{}
% Estilo de las subsecciones
\titleformat{\subsection}
{\color{fg_ssec}\huge\bfseries}
{\color{fg_ssec}\thesubsection}{1em}{}
% Estilo de las subsecciones
\titleformat{\subsubsection}
{\color{fg_sssec}\LARGE\bfseries}
{\color{fg_sssec}\thesubsubsection}{1em}{}

%%%%%%%%%%%%%%%%%%%%%%%%%%%%%%%%%%%%%%%%%%%%%%%%%%%%%%%%%%%%%%%%%%%%%%%%%%%%%%%
% MISCELÁNEO

\renewcommand{\contentsname}{Índice} % Cambiar el título del índice
\setlength\parindent{0pt} % Tamaño de la sangría


\begin{document}
%%%%%%%%%%%%%%%%%%%%%%%%%%%%%%%%%%%%%%%%%%%%%%%%%%%%%%%%%%%%%%%%%%%%%%%%%%%%%%%
% PORTADA

\begin{titlepage}
	\newpagecolor{bg_por} % Color de la portada
	\centering
	%\includegraphics[width=0.7\textwidth]{imagenes/ugr_logo.jpg} \\ % Logo
	\vspace{7em}
	\centering
	\color{fg_por}{
	%\title{Implementación de la técnica de mezclado de colores usando la ecuación del Poisson}
		\fontsize{50pt}{50pt}{\scshape{
		Técnica de Búsqueda Local\\ 
        y \\
        Algoritmos Greedy\\
        para el Problema del Aprendizaje \\
        de Pesos 
        en Características.}} % Título
	}
	\vfill
	\centering
	\color{fg_por}{\large{\autores}} \\
	\vspace{2em}
	\color{fg_por}{\Large{\textit{\fecha}}} \\
\end{titlepage}
\restorepagecolor

%%%%%%%%%%%%%%%%%%%%%%%%%%%%%%%%%%%%%%%%%%%%%%%%%%%%%%%%%%%%%%%%%%%%%%%%%%%%%%%
% ÍNDICE

\newpage
\tableofcontents
\newpage
\listoffigures
\clearpage

%%%%%%%%%%%%%%%%%%%%%%%%%%%%%%

%% Añadir aquí las respectivas entradas 
%%%%%%%%%%%%%%%%%%%%%%%%%%%%%%%%%%%%%%%%%%%%%%%%%%%
%%  Preprocesado de los datos 
%%%
%%%%%%%%%%%%%%%%%%%%%%%%%%%%%%%%%%%%%%%%%%%%%%%%%%%

\section{Preprocesado de los datos}

Tras leerse de los ficheros y comprobado que no presentan datos perdidos, se han separado los datos y etiquetas y normalizado los distintos atributos de cada dato. 

Hay un caso en que los valores son tan parecidos que la diferencia entre el mínimo y el máximo son iguales como criterio se ha dejado los datos a cero.


%%%%%%%%%%%%%%%%%%%%%%%%%%%%%%%%%%%%%%%%%%%%%%%%%%%
%%  Descripción del algoritmo de 1-NN
%%%%%%%%%%%%%%%%%%%%%%%%%%%%%%%%%%%%%%%%%%%%%%%%%%%
\section{ Descripción del algoritmo 1-NN}

Dado un conjunto de datos de entrenamiento etiquetados, se pretende construir un clasificador 
a partir de devolver la categoría más cercana a los datos ya conocidos. 

La implementación del algoritmo es directa dada la descripción, consistiendo en 

\begin{itemize}
    \item Almacenamiento adecuado de los datos de entrenamiento.
    \item Evaluación de la distancia.
\end{itemize}

\subsection{Algoritmo 1-NN}

Primero crearemos una estructura que almacena los datos y devuelve
una función dependiente de una función de error 
y el valor que se desea obtener.
% pseudo código cálculo cálculo de gradiente 

\begin{algorithm}
    \caption{Algoritmo de  1-NN}\label{euclid}
    \hspace*{\algorithmicindent} 
        \textbf{Entrada}:
        \begin{itemize}
            \item $x$: Vector a clasificar .
            \item $dist$: Distancia entre vectores $dist$.
            \item $data:$ Matriz  datos , cada columna un atributo distinto, cada fila los atributos del vector $i$-ésimo.
            \item $labels:$ Clase a la que pertenece cada dato (importa la posición).
        \end{itemize}
        \hspace*{\algorithmicindent} 
        \textbf{Salida}:
    Devuelve la clase a la que clasifica $x$.        
    \begin{algorithmic}[1]
      \Procedure{OneNN}{$x, dist, data,labels$}
          \State $distanciaMínima \gets\infty$
          \State $clase \leftarrow NADA$
        \For{\texttt{all} $(e_i, c_i) \in data \times labels$}
            \If{ $dist(e_i, x) < distanciaMínima$}
                \State $distanciaMínima \gets dist(e_i, x)$
                \State $clase  \gets c_i$
            \EndIf
        \EndFor
        \State \textbf{return} $clase$\Comment{Clase en la que se ha clasificado a $x$}
      \EndProcedure
    \end{algorithmic}
  \end{algorithm}



\end{document}