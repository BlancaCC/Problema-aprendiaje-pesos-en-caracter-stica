\documentclass[11pt,twoside,titlepage,a4paper]{book}

%%%%%%%%%%%%%%%%%%%%%%%%%%%%%%%%%%%%%%%%%%%%%%%%%%%%%%%%%%%%%%%%%%%%%%%%%%%%%%%
% PAQUETES
%salu2 - Valentino
\usepackage{xcolor} % Colores
\usepackage[xcolor]{mdframed} % Marcos
\usepackage{amsmath} % Matemáticas
\usepackage{amsfonts} % Letras caligráficas para matemáticas
\usepackage{mathtools} % Matemáticas extra
\usepackage{amsthm} % Teoremas
\usepackage{listingsutf8} % Código
\usepackage[a4paper]{geometry} % Márgenes
\usepackage{enumitem} % Opciones de personalización de listas
\usepackage{fancyhdr} % Encabezado / Pie de página
\usepackage{titlesec} % Títulos
\usepackage{pagecolor} % Colorear las portadas
\usepackage{graphicx} % Imágenes
\usepackage{hyperref} % Referencias
\usepackage{sidenotes} % Notas en el margen
\usepackage{pgfplots} % Gráficos de funciones
\usepackage{biblatex} % Bibliografía
\bibliography{bibliografia.bib}
\usepackage{caption}
\usepackage{subcaption}
% Para escribir pseudocódigo
%\usepackage{algorithm}  
%\usepackage{algorithmic} 
\usepackage{algorithm}% http://ctan.org/pkg/algorithm
\usepackage{algpseudocode}% http://ctan.org/pkg/algorithmicx
%%%%%%%%%%%%%%%%%%%%%%%%%%%%%%%%%%%%%%%%%%%%%%%%%%%%%%%%%%%%%%%%%%%%%%%%%%%%%%%
% COMANDOS PERSONALIZADOS

% Año o cualquier otra información para la portada
\newcommand{\fecha}{
\today
}
% Autores del documento
\newcommand{\autores}{
 Blanca Cano Camarero
}
% DNI
\newcommand{\DNI}{
 75577392J
}
% Email 
\newcommand{\email}{
 blancacano@correo.ugr.es
}
% Grupo de prácticas 
\newcommand{\grupo}{
 Grupo de prácticas 2, lunes.
}

\newcommand{\margenimagen}{
\newgeometry{
    left=2.5cm, % Margen izquierdo
	right=5cm, % Margen derecho
	bottom=2.5cm % Margen inferior}
}
}

%%%%%%%%%%%%%%%%%%%%%%%%%%%%%%%%%%%%%%%%%%%%%%%%%%%%%%%%%%%%%%%%%%%%%%%%%%%%%%%
% TIPOGRAFÍA

\usepackage{heuristica}
\usepackage[heuristica,vvarbb,bigdelims]{newtxmath}
\usepackage[T1]{fontenc}
\renewcommand*\oldstylenums[1]{\textosf{#1}}
\usepackage[spanish]{babel}

%%%%%%%%%%%%%%%%%%%%%%%%%%%%%%%%%%%%%%%%%%%%%%%%%%%%%%%%%%%%%%%%%%%%%%%%%%%%%%%
% DEFINICIÓN DE COLORES

% COLORES DE LA ESTRUCTURA DEL DOCUMENTO
%\definecolor{bg_por}{HTML}{8A0808} % Portada
\definecolor{bg_por}{HTML}{95b2b0} % Portada
\definecolor{fg_por}{HTML}{FFFFFF} % Texto de la portada
\definecolor{fg_sec}{HTML}{8A0808} % Títulos de las secciones
\definecolor{fg_ssec}{HTML}{610505} % Títulos de las subsecciones
\definecolor{fg_sssec}{HTML}{300303} % Títulos de las subsubsecciones
\definecolor{fg_head}{HTML}{610B0B} % Texto del encabezado
% COLORES PARA CÓDIGO
\definecolor{li_code}{HTML}{8A0808} % Línea a la izquierda
\definecolor{rw_code}{HTML}{610B0B} % Palabras reservadas
\definecolor{st_code}{HTML}{300303} % Cadenas de caracteres
\definecolor{cm_code}{HTML}{333333} % Comentarios
% COLORES PARA LISTAS
\definecolor{l_1}{HTML}{8A0808} % Primer símbolo
\definecolor{l_2}{HTML}{610505} % Primera indentación
\definecolor{l_3}{HTML}{300303} % Segunda indentación
\definecolor{l_4}{HTML}{000000} % Tercera indentación
% COLORES PARA MARCOS
\definecolor{li_ejs}{HTML}{8A0808} % Línea marco ejemplos
\definecolor{li_defs}{HTML}{610505} % Línea marco definiciones
\definecolor{bg_ejs}{HTML}{FFEDEE} % Fondo ejemplos
\definecolor{bg_defs}{HTML}{FFE0DF} % Fondo definiciones
% COLORES PARA REFERENCIAS
\definecolor{fg_url}{HTML}{610505} % Links


%%%%%%%%%%%%%%%%%%%%%%%%%%%%%%%%%%%%%%%%%%%%%%%%%%%%%%%%%%%%%%%%%%%%%%%%%%%%%%%
% REFERENCIAS

\hypersetup{
	pdftitle={	Memoria metaheurística}, % Título del pdf
	pdfauthor={}, % Autor del pdf
    colorlinks=true, % Referencias con color
    linkcolor=black, % Color de las referencias internas
    urlcolor=fg_url, % Color de los links
    citecolor=fg_ssec, % Color de las referencias
}
\urlstyle{same} % Links con el mismo tipo de letra

%%%%%%%%%%%%%%%%%%%%%%%%%%%%%%%%%%%%%%%%%%%%%%%%%%%%%%%%%%%%%%%%%%%%%%%%%%%%%%%
% TEOREMAS

\numberwithin{equation}{section} % Numeración de ecuaciones
% Teoremas-Lemas-Definiciones-Corolarios
\newtheoremstyle{usual} % Nombre del estilo
{} % Espacio por encima
{} % Espacio por debajo
{} % Estilo del cuerpo
{} % Indentación
{\bfseries} % Estilo de la cabecera
{} % Símbolo tras la cabecera
{ } % Espacio tras la cabecera
{\thmname{#1}\thmnumber{ #2 }\thmnote{(\textit{#3})}:} % Especificación de la cabecera
\theoremstyle{usual}
\newtheorem{theorem}{Teorema}[section] % Comando para los teoremas

%%%%%%%%%%%%%%%%%%%%%%%%%%%%%%%%%%%%%%%%%%%%%%%%%%%%%%%%%%%%%%%%%%%%%%%%%%%%%%%
% CÓDIGO

\lstset{
	basicstyle=\footnotesize\ttfamily, % Estilo del código
	inputencoding=utf8/latin1, % Codificación
	xleftmargin=1.3em, % Margen extra a la izquierda
	breaklines=true, % Romper líneas largas
	language=, % Lenguaje del código
	numbers=left, % Números de línea
	numbersep=8pt, % Separación de los números de línea
	tabsize=4, % Tamaño de los tabs
	frame=leftline, % Posición del enmarcado
	framerule=1pt, % Grosor del enmarcado
	showstringspaces=false, % Mostrar los espacios en las cadenas de caracteres
	keywordstyle=\color{rw_code}, % Estilo de las palabras reservadas
	numberstyle=\normalfont, % Estilo de los números de línea
	rulecolor=\color{li_code}, % Estilo del enmarcado
	commentstyle=\color{cm_code}, % Estilo de los comentarios
	stringstyle=\color{st_code} % Estilo de las cadenas de caracteres
}

%%%%%%%%%%%%%%%%%%%%%%%%%%%%%%%%%%%%%%%%%%%%%%%%%%%%%%%%%%%%%%%%%%%%%%%%%%%%%%%
% MÁRGENES

\geometry{
	left=2.5cm, % Margen izquierdo
	right=2.5cm, % Margen derecho
	bottom=2.5cm % Margen inferior
}

%%%%%%%%%%%%%%%%%%%%%%%%%%%%%%%%%%%%%%%%%%%%%%%%%%%%%%%%%%%%%%%%%%%%%%%%%%%%%%%
% LISTAS/TABLAS

\renewcommand{\arraystretch}{1.3} % Tamaño entre líneas de una tabla
% SÍMBOLOS LISTAS
\renewcommand{\labelitemi}{\color{l_1}$\bullet$} % Primer símbolo
\renewcommand{\labelitemii}{\color{l_2}$\circ$} % Símbolo primera indentación
\renewcommand{\labelitemiii}{\color{l_3}$\diamond$} % Símbolo segunda indentación
\renewcommand{\labelitemiv}{\color{l_4}$-$} % Símbolo tercera indentación
% SÍMBOLOS ENUMERACIONES
\renewcommand{\labelenumi}{\color{l_1}\bfseries\arabic{enumi}.} % Primer símbolo
\renewcommand{\labelenumii}{\color{l_2}\bfseries\Roman{enumii}.} % Símbolo primera indentación
\renewcommand{\labelenumiii}{\color{l_3}\bfseries(\alph{enumiii})} % Símbolo segunda indentación
\renewcommand{\labelenumiv}{\color{l_4}\bfseries\Alph{enumiv}.} % Símbolo tercera indentación
% DESCRIPCIONES
\renewcommand{\descriptionlabel}[1]{\hspace{\labelsep}\color{l_1}\textbf{#1}} % Color y estilo del título de la descripción

%%%%%%%%%%%%%%%%%%%%%%%%%%%%%%%%%%%%%%%%%%%%%%%%%%%%%%%%%%%%%%%%%%%%%%%%%%%%%%%
% ENCABEZADO/PIE DE PAGINA

\setlength{\headheight}{14pt} % Tamaño del encabezado
\pagestyle{fancy}
\fancyhf{}
% Para que aparezca el título de la sección y no el número 
\renewcommand{\sectionmark}[1]{%
\markboth{#1}{}}
% Encabezado
\fancyhead[LE,RO]{\color{fg_head}{\leftmark}} % A la izquierda en pares, derecha en impares
\fancyhead[RE,LO]{\color{fg_head}{}} % A la derecha en pares, izquierda en impares
% Pie de página
\fancyfoot[LE,RO]{\Large\textbf{\thepage}} % A la izquierda en pares, derecha en impares
\renewcommand{\headrulewidth}{0.5pt} % Grosor de la línea

%%%%%%%%%%%%%%%%%%%%%%%%%%%%%%%%%%%%%%%%%%%%%%%%%%%%%%%%%%%%%%%%%%%%%%%%%%%%%%%
% TÍTULOS

% Estilo de las secciones
\titleformat{\section}
{\color{fg_sec}\Huge\bfseries}
{\color{fg_sec}\thesection}{1em}{}
% Estilo de las subsecciones
\titleformat{\subsection}
{\color{fg_ssec}\huge\bfseries}
{\color{fg_ssec}\thesubsection}{1em}{}
% Estilo de las subsecciones
\titleformat{\subsubsection}
{\color{fg_sssec}\LARGE\bfseries}
{\color{fg_sssec}\thesubsubsection}{1em}{}
%%%%%%%%%%%%%%%%%%%%%%%%%%%%%%%%%%%%%%%%%%%%%%%%%%%%%%%%%%%%%%%%%%%%%%%%%%%%%%%
% Atajos


\newcommand{\R}{\mathbb{R}}
%%%%%%%%%%%%%%%%%%%%%%%%%%%%%%%%%%%%%%%%%%%%%%%%%%%%%%%%%%%%%%%%%%%%%%%%%%%%%%%
% MISCELÁNEO

\renewcommand{\contentsname}{Índice} % Cambiar el título del índice
\setlength\parindent{0pt} % Tamaño de la sangría


\begin{document}
%%%%%%%%%%%%%%%%%%%%%%%%%%%%%%%%%%%%%%%%%%%%%%%%%%%%%%%%%%%%%%%%%%%%%%%%%%%%%%%
% PORTADA

\begin{titlepage}
	\newpagecolor{bg_por} % Color de la portada
	\centering
	%\includegraphics[width=0.7\textwidth]{imagenes/ugr_logo.jpg} \\ % Logo
	\vspace{7em}
	\centering
	\color{fg_por}{
	%\title{Implementación de la técnica de mezclado de colores usando la ecuación del Poisson}
		\fontsize{50pt}{50pt}{\scshape{
		Metaheurísticas para el problema del Aprendizaje \\
        de Pesos 
        en Características.}} % Título
	}
	\vfill
	\centering
	\color{fg_por}{\large{\autores}} \\
	\vspace{2em}
	\color{fg_por}{\large{DNI: \DNI}} \\
	\vspace{2em}
	\color{fg_por}{\large{correo: \email}} \\
	\vspace{2em}
	\color{fg_por}{\large{\grupo}} \\
	\vspace{2em}
	\color{fg_por}{\Large{\textit{\fecha}}} \\
\end{titlepage}
\restorepagecolor

%%%%%%%%%%%%%%%%%%%%%%%%%%%%%%%%%%%%%%%%%%%%%%%%%%%%%%%%%%%%%%%%%%%%%%%%%%%%%%%
% ÍNDICE

\newpage
\tableofcontents
\newpage
\listoffigures
\clearpage

%%%%%%%%%%%%%%%%%%%%%%%%%%%%%%

%% Añadir aquí las respectivas entradas 
%%%%%%%%%%%%%%%%%%%%%%%%%%%%%%%%%%%%%%%%%%%%%%%%%%%%%%%%%%%%%
%%% DESCRIPCIÓN DEL PROBLEMA 
%%%%%%%%%%%%%%%%%%%%%%%%%%%%%%%%%%%%%%%%%%%%%%%%%%%%%%%%%%%%%

\section{Descripción del problema}  

El problema de Aprendizaje de Pesos en Características (APC) es un problema de aprendizaje en clasificación para el cual se disponen de una muestra de objetos $\{w_i\}_{1 \leq i \leq N}$ 
representados en función de sus valores en una serie de atributos dados por $x$:
\begin{equation*}
    w_i \text{ tiene asociado el vector de $t$ atributos } (x_1(w_i), \ldots, x_t(w_i)).
\end{equation*}
Cada objeto pertenece a una de las $m$ clases existentes 
$\{C_1, \ldots, C_m\}$.

El objetivo de nuestro problema es poder clasificar cualquier $w$ de un modo automático.

Trataremos de afrontar el problema mediante el algoritmo de clasificación 1-NN, es decir clasificar cierto w a partir de la etiqueta del vector más cercano a él que se conozca.
\chapter{Práctica 1: Técnicas de Búsqueda Local 
y 
Algoritmos Greedy}
%%%%%%%%%%%%%%%%%%%%%%%%%%%%%%%%%%%%%%%%%%%%%%%%%%%
%%  Preprocesado de los datos 
%%%
%%%%%%%%%%%%%%%%%%%%%%%%%%%%%%%%%%%%%%%%%%%%%%%%%%%

\section{Preprocesado de los datos}

Tras leerse de los ficheros y comprobado que no presentan datos perdidos, se han separado los datos y etiquetas y normalizado los distintos atributos de cada dato. 

Hay un caso en que los valores son tan parecidos que la diferencia entre el mínimo y el máximo son iguales como criterio se ha dejado los datos a cero.


%%%%%%%%%%%%%%%%%%%%%%%%%%%%%%%%%%%%%%%%%%%%%%%%%%%
%%  Descripción del algoritmo de 1-NN
%%%%%%%%%%%%%%%%%%%%%%%%%%%%%%%%%%%%%%%%%%%%%%%%%%%
\section{ Descripción del algoritmo 1-NN}

Dado un conjunto de datos de entrenamiento etiquetados, se pretende construir un clasificador 
a partir de devolver la categoría más cercana a los datos ya conocidos. 

La implementación del algoritmo es directa dada la descripción, consistiendo en 

\begin{itemize}
    \item Almacenamiento adecuado de los datos de entrenamiento.
    \item Evaluación de la distancia.
\end{itemize}

\subsection{Algoritmo 1-NN}

Primero crearemos una estructura que almacena los datos y devuelve
una función dependiente de una función de error 
y el valor que se desea obtener.
% pseudo código cálculo cálculo de gradiente 

\begin{algorithm}
    \caption{Algoritmo de  1-NN}\label{euclid}
    \hspace*{\algorithmicindent} 
        \textbf{Entrada}:
        \begin{itemize}
            \item $x$: Vector a clasificar .
            \item $dist$: Distancia entre vectores $dist$.
            \item $data:$ Matriz  datos , cada columna un atributo distinto, cada fila los atributos del vector $i$-ésimo.
            \item $labels:$ Vector de clases a la que pertenece  cada dato (elemento i-ésimo corresponde al dato iésimo).
        \end{itemize}
        \hspace*{\algorithmicindent} 
        \textbf{Salida}:
    Devuelve la clase a la que clasifica $x$.        
    \begin{algorithmic}[1]
      \Procedure{OneNN}{$x, dist, data,labels$}
          \State $distanciaMínima \gets\infty$
          \State $clase \leftarrow NADA$
        \For{\texttt{all} $(e_i, c_i) \in data \times labels$} \Comment{Recorremos cada vector de atributo con su clase asociada}
            \If{ $dist(e_i, x) < distanciaMínima$}
                \State $distanciaMínima \gets dist(e_i, x)$
                \State $clase  \gets c_i$
            \EndIf
        \EndFor
        \State \textbf{return} $clase$\Comment{Clase en la que se ha clasificado a $x$}
      \EndProcedure
    \end{algorithmic}
  \end{algorithm}  

\subsection{Resultados para el 1-NN sin pesos} 

  Para la distancia euclidea sin ninguna ponderación de atributos, es decir, con 
  $w = (1,\ldots, 1)$ se han obtenido los siguientes resultados: 
  
 % \resizebox{\textwidth}{!}{

  \begin{table}[h]
    \centering
    \resizebox{\columnwidth}{!}{%
      \begin{tabular}{|c|c|c|c|c|c|c|}
     \hline
     & \multicolumn{2}{|c|}{\textit{Ionosphere}} 
     & \multicolumn{2}{|c|}{\textit{Parkinsons}} 
     &\multicolumn{2}{|c|}{\textit{Spectf-heart}}\\
     \hline
       & Clasificación  & Tiempo (ms)
       & Clasificación	& Tiempo (ms)
       & Clasificación	& Tiempo (ms)
       \\ \hline
Partición 1 
    & 84.507  & 2.258
    &	94.872 &	0.529
    & 84.286	& 2.408
\\
Partición 2 
    & 87.143   & 2.238
    & 100.0	 & 9.314
    & 78.571	& 9.923

\\
Partición 3 
    & 80.0  & 14.370
    & 94.872 &	0.592 
    & 81.429 &	2.729

\\
Partición 4 
  & 88.571  & 1.929
  & 92.308 &	0.615 
  & 85.714 &	2.695

\\
Partición 5 
  & 90.0  & 2.323
  & 97.4359 &	0.513
  & 82.609 &	2.787

\\ \hline
Medias 
  & 86.0442  & 4.623
  &	95.897 & 2.312
  & 82.522 & 4.109

\\ \hline
Desviación típica 
  & 3.941  & 3.743
  &	2.923 & 0.023
  & 2.745 & 3.637
\\
  \hline        
      \end{tabular}
  }
      \caption{Métricas de clasificación y tiempo  para el algoritmo 1-NN sin ponderación de pesos en los diferentes conjuntos de datos}
      \label{Tabla:1nn-base}
\end{table}

Utilizaremos los resultado de la tabla \ref{Tabla:1nn-base} como base para comparar la bondad de otras técnicas.

\subsubsection*{Reflexiones}  

A la vista de los resultados en clasificación en las tres bases de datos, con más de un $80\%$ de
 precisión y desviaciones típicas de menos de $4\%$, 
podría uno dejarse llevar por la particularidad y pensar que el algoritmo $1-$NN es bastante robusto 
y produce buenos resultados. Sin embargo estos no dejan de ser casos concretos y existirán conjuntos para el que falle, de hecho es fácil pensar un ejemplo: 

Consideramos $esPar:\mathbb{N} \longrightarrow \{0,1\}$
la regla subyacente que clasifica a nuestros datos y consideramos el conjunto de datos 
los conjuntos de entrenamiento son los ciclos generados por 
$<2>+2 = \{2^1, 2^2, 2^3, \ldots\}$ cuya etiqueta es $1$ y por otra parte
$<11>+11 = \{11, 11^2, 11^3, \ldots\}$ cuya etiqueta claramente es $0$.
Esta claro que existe el mismo número de números pares que impares, pero ante estos conjuntos es más probable que se clasifique como par.




\section{Algoritmo de búsqueda local}  

Si como distancia en el problema anterior tomamos 
\begin{equation}
    d_e(e_1, e_2)
     = 
     \sqrt{
         \sum_i
             w_i \cdot (e^i_1 - e^i_2)^2  
        +
        \sum_j w_j \cdot d_h(e_1^j, e_2^j)
         }
\end{equation}

El vector de pesos $w \in [0,1]^n$ es nuestra incógnita
y deberemos de encontrar el que mejore el clasificador de ponderación 1.

\subsection{Componentes del algoritmo de búsqueda local}  

Las métricas a observar de nuestro algoritmo son: 

\begin{itemize}
    \item \textbf{Precisión} o \texttt{tasa-clasificación} Rendimiento promedio para $k=1$ y utilizado \textit{leave one out}.
    \item \textbf{Tasa de reducción}, número de características que se consideran como clasificador. 
    \begin{equation}
        \text{tasa-reducción} 
        = 
        100
        \frac{\text{número de }w_i < 0.1}{\text{número de características}}.
    \end{equation}
    \item \textbf{Función de evaluación} Permite cuantificar el éxito de nuestra selección de pesos de acorde a una  combinación de precisión y simplicidad, su expresión viene dada por 
    \begin{equation}
        F(w) = 
            \alpha \texttt{tasa-clasificación}
            +
            (1 - \alpha) \texttt{tasa-reducción}.
    \end{equation}
    Nótese que $\alpha$ es la ponderación de relevancia que se le da al modelo.
\end{itemize}

\subsection{Búsqueda local del primero mejor}  

El algoritmo que vamos a usar es el conocido como el del \textit{primero mejor} y que radica en esencia de que  cuanto se genera 
una solución que mejora a la actual se aplica el movimiento y se pasa a la siguiente iteración.

Se detiene la búsqueda cuando se haya generado un número máximo de vecinos que no mejora el resultado.

Descripción del algoritmo, necesitamos primero tener una función para general vecinos, de acorde a los requisitos, esta será 
la función \texttt{GeneraVecino($w,\sigma$)} que devolverá un vector 

\textcolor{red}{El vector no puede tomar valores negativos} y además serán menor que 1.
\begin{algorithm}
    \caption{Búsqueda local del primero mejor}
    \hspace*{\algorithmicindent} 
        \textbf{Entrada}:
        \hspace*{\algorithmicindent} 
        \textbf{Salida}:
        Vector de pesos $w$.        
    \begin{algorithmic}[1]
      \Procedure{PrimeroMejor}{numeroMáximoVecinos, evaluacionesMáximas}
            \Comment{Inicializamos pesos}
          \State $w \gets$ vector aleatorio  
          \State $iteraciones \gets 0$ \Comment{Indica número de generaciones}
        \While{iteraciones < iteracionesMáximas}

        \EndWhile
        \State \textbf{return} $clase$\Comment{Clase en la que se ha clasificado a $x$}
      \EndProcedure
    \end{algorithmic}
  \end{algorithm}




%%%%%%%%%%%%%%%%%%%%%%%%%%%%%%%%%%%%
%%% Algoritmo Greedy 
%%%%%%%%%%%%%%%%%%%%%%%%%%%%%%

\section{Algoritmo Greedy}

Con  el fin de reducir el costo computacional 
introducido por el algoritmo anterior vamos a proceder a un algoritmo voraz denominado 
\textit{RELIEF}.

La idea que subyace es la siguiente: ponderaremos los atributos que son relevantes 
para clasificar elementos, es decir aquellos que marquen diferencias entre elementos que clasifiquen igual \textit{amigos} y aquellos que clasifiquen diferente \textit{enemigos}.

\begin{algorithm}[H]
    \caption{Algoritmo RELIEF}
        \hspace*{\algorithmicindent} 
        \textbf{Salida}:
        Vector de pesos $w$.        
    \begin{algorithmic}[1]
      \Procedure{RELIEF}{$datos$,
      $etiquetas$
    }
    \Comment{Almacenamos las distancias y si son de la misma clase}

    \For{ para cada dos pares (atributo, etiqueta)  $(x_i,y_i),(x_j,y_j)$
    tal que $i < j$
    } 
          \State $distancia[i,j] \gets (DistanciaEuclidea(x_i,x_j), y_i == y_j) $ 
    \EndFor

    \Comment{Inicializamos vector que pondera los atributos a cero}
    \State $w \gets vectorDeCeros$ 

    \Comment{Inicializamos variables auxiliares}
    \State $distanciaEnemigo \gets \infty$
    \State $distanciaAmigo \gets \infty$
    \State $indiceEnemigo \gets \emptyset$
    \State $indiceAmigo \gets \emptyset$
    
    \Comment{Encontramos los enemigos y amigos más cercanos }
    \For{ para cada dato del $x_i$ conjunto de entrenamiento 
    } 
        \For{ para cada dato del $x_j$ conjunto de entrenamiento distinto de $x_i$
        } 
            \State $a \gets minimo\{i,j\}$
            \State $b \gets maximo\{i,j\}$

            \Comment{Caso en que son amigos}
            \If{$y_i == y_j$}
                \If{ $distancia[a,b] < distanciaAmigo$ }
                    \State $distanciaAmigo \gets distancia[a,b]$
                    \State $indiceAmigo \gets j$
                \EndIf
            \EndIf
            \Comment{Caso en que son enemigos}
            \If{$y_i \neq y_j$}
                \If{ $distancia[a,b] < distanciaEnemigo$ }
                    \State $distanciaAmigo \gets distancia[a,b]$
                    \State $indiceEnemigo \gets j$
                \EndIf
            \EndIf
            
            \Comment{Añadimos peso a las diferencia entre amigos y enemigos}
            \State $w \gets w + |x_i - x_{indiceAmigo}| + |x_i - x_{indiceEnemigo}|$
        \EndFor
        
        \State $w \gets PonemosValoresNegativosACero(w)$
        \Comment{Truncamos los valores 
        negativos a cero}
        \State $w \gets Reescalamos(w)$
        \Comment{Dividimos cada coeficiente por el máximo}
    \EndFor

    \State \textbf{return} $w$
      \EndProcedure
    \end{algorithmic}
  \end{algorithm}

  Podemos ver como al tener el algoritmo greedy 
una eficiencia $\mathcal{O}(n^2)$, pero de constantes ocultas mucho menores que el algoritmo de búsqueda local. 

\subsection{Los resultados obtenidos han sido los siguientes}


\subsubsection{Ionosphere} 

\begin{table}[H]
  \centering
  \begin{tabular}{|c|c|c|c|c|}
    \hline
    & \multicolumn{4}{|c|}{\textit{Ionosphere}}  \\
    \hline
    &	Clasificación &		Reducción	
    &	Agregación	&	Tiempo en ms \\
    \hline
    Partición 1	&  87.324  &  2.941  &  45.133  &  60.649  \\
    Partición 2 &	 87.143  &  2.941  &  45.042  &  46.026  \\
    Partición 3 &  75.714  &  2.941  &  39.328  &  32.651  \\
    Partición 4	&  84.286  &  2.941  &  43.613  &  61.079 \\
    Partición 5	&  84.286  &  2.941  &  43.613  &  44.848  \\
    \hline
    Medias 	 &  83.751  &  2.941  &  43.346  &  49.051 \\
    \hline
    Desviación típica &	 4.728  &  0.0  &  2.364  &  11.989  \\ 
    \hline  
  \end{tabular}
  \caption{Resultados búsqueda greedy para los datos \textbf{ionosphere}}
  \label{table:greedy_ionosphere}
\end{table}

Se han analizado a demás los pesos resultantes de cada partición, obteniendo con ello: 
Un vector de pesos medio con un redondeo de tres decimales es 

\begin{align*}
w_{medio} = [  
&  0.285, 0.0, 0.343, 0.355, 0.321, 0.482, 0.44, 0.974, 0.577, 0.471, 0.621, 0.544, 0.841, 0.706  \\
&  0.706, 0.842, 0.515, 0.919, 0.467, 0.848, 0.546, 0.915, 0.652, 0.801, 0.574, 0.767, 0.477, 0.604  \\
&  0.604, 0.572, 0.72, 0.485, 0.772, 0.537, 0.744, 0.573 
 ]
\end{align*}

Donde cada coeficiente del vector presenta una desviación típica media de 

\begin{align*}
  w_{desv. tip.} = [ 
    &  0.085, 0.0, 0.048, 0.027, 0.059, 0.011, 0.012, 0.058, 0.127, 0.104, 0.083, 0.029, 0.031, 0.01  \\
    &  0.01, 0.065, 0.08, 0.045, 0.03, 0.085, 0.031, 0.037, 0.016, 0.016, 0.139, 0.051, 0.058, 0.026  \\
    &  0.026, 0.08, 0.041, 0.117, 0.038, 0.099, 0.002, 0.077 
   ]
  \end{align*}

  La media de de las coeficientes de la desviación típica es de $0.053$, esto es un valor bastante bueno y significa que solemos converger a soluciones de $w$ similares. 
  Observemos además que podemos reducir un atributo, el segundo con bastante certeza, ya que por su desviación típica vemos que siempre se reduce. 




  \subsubsection{Parkinsons} 

  \begin{table}[H]
    \centering
    \begin{tabular}{|c|c|c|c|c|}
      \hline
      & \multicolumn{4}{|c|}{\textit{Parkinsons}}  \\
      \hline
      &	Clasificación &		Reducción	
      &	Agregación	&	Tiempo en ms \\
      \hline
      Partición 1	&    94.872  &  0.0  &  47.436  &  17.603   \\
      Partición 2 &	   100.0  &  0.0  &  50.0  &  7.515   \\
      Partición 3 &    94.872  &  0.0  &  47.436  &  17.254    \\
      Partición 4	&    92.308  &  0.0  &  46.154  &  17.297  \\
      Partición 5	&    94.872  &  0.0  &  47.436  &  7.449   \\
      \hline
      Medias 	 &    95.385  &  0.0  &  47.692  &  13.424   \\
      \hline
      Desviación típica &	   2.809  &  0.0  &  1.404  &  5.426   \\ 
      \hline  
    \end{tabular}
    \caption{Resultados búsqueda greedy para los datos \textbf{Parkinson}}
    \label{table:greedy_parkinson}
  \end{table}
  
  Se han analizado a demás los pesos resultantes de cada partición, obteniendo con ello: 
  Un vector de pesos medio con un redondeo de tres decimales es 
  El vector de pesos medio es 

\begin{align*}
  w_{medio} = [ 
    & 0.601, 0.347, 0.659, 0.346, 0.257, 0.317, 0.323, 0.316, 0.462, 0.424, 0.538, 0.422, 0.35, 0.538 \\
    & 0.538, 0.225, 0.558, 0.979, 0.675, 0.528, 0.695, 0.681, 0.521]
  \end{align*}
  
  Donde cada coeficiente del vector presenta una desviación típica media de 
  
  \begin{align*}
    w_{desv. tip.} = [ 
      & 0.029, 0.019, 0.191, 0.051, 0.002, 0.061, 0.05, 0.061, 0.082, 0.088, 0.098, 0.079, 0.07, 0.098 \\
      & 0.098, 0.03, 0.086, 0.048, 0.145, 0.031, 0.057, 0.179, 0.003
     ]
\end{align*}

El valor de peso medio es de $0.891$ y la desviación típica de $0.181$, luego por media todos los pesos son por lo general relevantes. 

Además la media de las desviaciones típicas es de $0.071$ lo que indica que se converge con gran robustez a la misma solución.



\subsubsection{Spectf Heart} 

\begin{table}[H]
  \centering
  \begin{tabular}{|c|c|c|c|c|}
    \hline
    & \multicolumn{4}{|c|}{\textit{Spectf Heart}}  \\
    \hline
    &	Clasificación &		Reducción	
    &	Agregación	&	Tiempo en ms \\
    \hline
    Partición 1	&    84.286  &  0.0  &  42.143  &  63.055    \\
    Partición 2 &	  80.0  &  0.0  &  40.0  &  55.255    \\
    Partición 3 &       80.0  &  0.0  &  40.0  &  44.498     \\
    Partición 4	&      90.0  &  0.0  &  45.0  &  62.065    \\
    Partición 5	&     85.507  &  0.0  &  42.754  &  55.167    \\
    \hline
    Medias 	 &    83.959  &  0.0  &  41.979  &  56.008   \\
    \hline
    Desviación típica &	   4.194  &  0.0  &  2.097  &  7.418    \\ 
    \hline  
  \end{tabular}
  \caption{Resultados búsqueda greedy para los datos \textbf{Spectf Heart}}
  \label{table:greedy_Spectf Heart}
\end{table}

Se han analizado a demás los pesos resultantes de cada partición, obteniendo con ello: 
Un vector de pesos medio con un redondeo de tres decimales es 
El vector de pesos medio es 

\begin{align*}
w_{medio} = [ 
    &   0.589, 0.953, 0.587, 0.782, 0.562, 0.656, 0.407, 0.609, 0.455, 0.641, 0.416, 0.513, 0.586, 0.643  \\
    &  0.643, 0.614, 0.592, 0.458, 0.484, 0.689, 0.537, 0.406, 0.526, 0.556, 0.651, 0.822, 0.832, 0.727  \\
    &  0.727, 0.669, 0.736, 0.898, 0.614, 0.343, 0.502, 0.555, 0.517, 0.536, 0.697, 0.59, 0.537, 0.77  \\
    &  0.77, 0.592, 0.709, 0.738, 0.903]
\end{align*}

Donde cada coeficiente del vector presenta una desviación típica media de 

\begin{align*}
  w_{desv. tip.} = [ 
    &  0.039,0.105, 0.01, 0.088, 0.0, 0.034, 0.016, 0.065, 0.072, 0.09, 0.094, 0.03, 0.085, 0.199  \\
    &  0.199, 0.021, 0.008, 0.08, 0.064, 0.095, 0.059, 0.053, 0.012, 0.157, 0.109, 0.179, 0.122, 0.214  \\
    &  0.214, 0.129, 0.014, 0.041, 0.133, 0.019, 0.087, 0.019, 0.067, 0.058, 0.091, 0.002, 0.048, 0.028  \\
    &  0.028, 0.06, 0.037, 0.12, 0.023 
   ]
\end{align*}

El valor de peso medio es de $0.618$ y la desviación típica de $0.139$, luego por media todos los pesos son por lo general relevantes. 

Además la media de las desviaciones típicas es de $0.070$ lo que indica que se converge con gran robustez a la misma solución.

\subsubsection*{Reflexiones}

Como podemos observar la reducción ha sido prácticamente nula, esto es algo totalmente 
normal ya que no se ha tenido en cuenta durante el diseño del algoritmo.

Volviendo a utilizar las comparativas de tiempo 
de acorde a la métrica (\ref{metricaNuevaTiempo}) se tiene que 

para Ionosphere ha aumentado el tiempo un $1061.021 \%$, para 
Parkinson $578.37 \%$ y finalmente 
Specf heart $1363.056 \%$  


%%%%%%%%%%%%%%%%%%%%%%%%%%%%%%%%%%%%%%%%%%
%% Valoraciones de todos los resultados
%%%%%%%%%%%%%%%%%%%%%%%%%%%%%%%%%%%%%%%%%%      

\section{Comparativa entre los distintos algoritmos}

Todos estamos 
A continuación expondremos una recopilación de la tasa de clasificación media, reducción media  y tiempo medio.


\begin{table}[h]
    \centering
    \resizebox{\columnwidth}{!}{%
      \begin{tabular}{|c|c|c|c|c|c|c| c|c|c| }
     \hline
     & \multicolumn{3}{|c|}{\textit{Ionosphere}} 
     & \multicolumn{3}{|c|}{\textit{Parkinsons}} 
     &\multicolumn{3}{|c|}{\textit{Spectf-heart}}\\
     \hline
       Algoritmo
       & Clasificación  & Tiempo (ms) & Red.
       & Clasificación	& Tiempo (ms) & Red.
       & Clasificación	& Tiempo (ms) & Red. 
       \\ \hline

    1NN base
    & 86.0442  & 4.623 & 0
    &	95.897 & 2.312 & 0
    & 82.522 & 4.109 & 0
    \\ \hline

    Búsqueda local
    & 85.191 & 154672.748 & 27.06
    & 92.821 & 41035.95 & 23.64
    & 82.807 & 357893.84 & 20.45
    \\ \hline

    Greedy
    & 82.807 & 357893.84 & 0
    & 83.751 & 49.051 & 2.941
    & 83.959 & 56.01 & 0
    \\ \hline
      
    \end{tabular}
    }
    \caption{Comparativas tasa de clasificación media, reducción media  y tiempo medio distintos algoritmos.}
    \label{Tabla:comparativas final}
\end{table}

Para poder comparar la bondad estableceremos primero los criterios 
y respecto a qué. 

Si el objetivo de nuestro problema era reducir el número de parámetros de acorte
a cierto compromiso entre tasa de clasificación y reducción \textit{fitness}, está claro 
que la única alternativa posible es utilizar búsqueda local.

Ya que suponer que el algoritmo \textit{RELIEF} reduciría era demasiado optimista hasta para ser una heurística.

Por su parte si lo que nos interesa es una clasificación buena con un compromiso de tiempo a la vista de estos casos particulares podemos afirmar que 1NN está ofreciendo buenos resultados. 

Y de hecho en el caso de \textit{iosphere} y \textit{Parkinsons} llega incluso a superar al algoritmo voraz, poniendo de manifiesto su patología de no encontrar óptimos. 

Sin embargo, si lo que nos interesase fuera la clasificación sin importar el tiempo
mi propuesta última sería utilizar el algoritmo de búsqueda local con una función objetivo que busque maximizar la clasificación sin tener en cuenta la reducción de dimensionalidad.

Concluimos también que en estos casos particulares, tras observar los pesos encontrados, el único algoritmo estable con el que poder establecer hipótesis para un estudio estadístico de la relevancia de ciertos atributos es el greedy.



%%%%%%%%%%%%%%%%%%%%%%%%%%%%%%%%%%%%%%
%% Estructura 
%%%%%%%%%%%%%%%%%%%%%%%%%%%%%%%

\section{Estructura }

Los ficheros que se entregan son los siguientes: 
\begin{itemize}
    \item \textbf{Instancias APC}: Ficheros con datos.
    \item \textbf{algoritmos búsqueda}: Generar vecinos. 
    \item \textbf{learner}: Abstrae algoritmos de búsqueda local.
    \item \textbf{resultados}: Carpeta con csv de los resultados y los ficheros que los generan.
    \item \textbf{utils}: Funciones auxiliares útiles.
\end{itemize}

Para ejecutarlo basta con ejecutar \texttt{make result}.

Además todos los algoritmos se han ejecutado con semilla $0$. 

\section{Metodología}  

Se ha desarrollado la práctica con una metodología ágil que puede seguirse 
en el repositorio de github \url{https://github.com/BlancaCC/Problema-aprendizaje-pesos}.

El lenguaje utilizado ha sido Julia y además se han hecho implementaciones con código en paralelo 
(el cual no se contabiliza en las mediciones).

% Práctica 2: 
\chapter{Práctica 2: Técnicas de Búsqueda basadas en Poblaciones}
% Esquema de las soluciones empleadas

% Descripción en pseudocódigo de la función objetivo 

% Pseudo código del proceso de generación de soluciones aleatorias

% Pseudocódigo de los AGs y los operadores de cruce y mutación.
%%%%%%%%%%%%%%%%%%%
%% Operadores de cruce 
%%%%%%%%%%%%%%%%%

\section{Operadores de cruce}  

\subsection{ Operador de cruce BLX$-\alpha$}  

El pseudo código del operador de cruce creado es 

% Operador de cruce 
\begin{algorithm}[H]
    \caption{Operador de cruce BLX$-\alpha$}
    \hspace*{\algorithmicindent} 

        \textbf{Entrada}:
        \begin{itemize}
          \item $C_1 = (c_{1 1}, \ldots, c_{1 n}), C_2=(c_{2 1}, \ldots, c_{2 n})$ 
          dos vectores de dimensión $n$, que son los cromosomas. 
          \item  $\alpha \in [0,1]$.
        \end{itemize}
        
        \hspace*{\algorithmicindent} 

        \textbf{Salida}:
        Dos vectores $H_1, H_2$ de tamaño $n$ que son los descendientes.

    \begin{algorithmic}[1]
        % Para cada uno de los vectores
        \For{ $i \in \{1, \ldots, n\}$}
              \begin{align*}
                & c_{\max} \gets \max\{ c_{1 i}, c_{2 i}\} \\
                & c_{\min} \gets \min\{ c_{1 i}, c_{2 i}\} \\
                & I \gets c_{\max} - c_{\min} 
              \end{align*}
          
              \For{$k \in \{1,2\}$}
              $$h_{k j} \gets
              \text{ valor aleatorio perteneciente a } 
             [c_{min} - I \alpha, c_{max} + I \alpha]$$
           \EndFor 
           \EndFor 
          \For{$k \in \{1,2\}$}
             $$H_k \gets (h_{k 1}, \ldots, h_{k n}) $$
          \EndFor 
          
       \State \textbf{return} $H_1, H_2$
    \end{algorithmic}
  \end{algorithm}

 
%%%%%%%%%%%%%%%%%%%%%%%%%%%%%%%%%%%%%%%%%%%%%%%%%%%%%%%%
%% Algoritmos genéticos  generalistas 
%%
%%%%%%%%%%%%%%%%%%%%%%%%%%%%%%%%%%%%%%%%%%%%%%%%%%%%%%%%%

\section{ Algoritmos genéticos}
\subsection{Algoritmos genéticos generalistas}  
Para realizar esta solución hemos planteado el siguiente esquema. 

\begin{enumerate}
    \item Inicializamos la primera generación $t=1$ con $M = 30$ cromosomas que son
    vectores inicializado de manera aleatoria en $\{0\} \cup [0.1; 1]$ de tamaño $d$  donde $d$ es el número de atributos a analizar.  
    \item  Evaluamos la función objetivo de estos resultados. 
    \item Procedemos a realizar las sucesivas generaciones hasta que se cumpla el criterio de parada. 
\end{enumerate}

La generación de las sucesivas generaciones consiste en el siguientes algoritmo que tiene de  
\textbf{Entrada}:
        \begin{itemize}
          \item \texttt{EMFE}: evaluaciones máximas función evaluación.
          \item $M \in \mathbb{N}$: número de cromosomas en cada generación. 
          \item $P_c \in [0,1]$: probabilidad de cruce. 
          \item $P_m \in [0,1]$: probabilidad de mutación. 
          \item $d$ tamaño de cada cromosoma. 
          \item $F$ función de evaluación. 
          \item \textit{Función-cruce}$: C \times C \longrightarrow C \times C$ Algoritmo de cruce. 
          \item \textit{Función-Mutación}$: C \longrightarrow C $ Función de mutación. 
        \end{itemize}

y de 
\textbf{Salida}:Última generación de cromosomas a la que se ha llegado,
        consiste en un conjunto de tamaño $M$ de pares de cromosomas y su valor en la función de evaluación.

% Algoritmo genético generacional  
\begin{algorithm}[H]
    \caption{Algoritmo genético generacional}

    \begin{algorithmic}[1]
        \State Generamos primera generación \\
        \begin{align*}
            \textit{Generación }\gets & \text{ conjunto de tamaño $M$ tal que } \\
            \{ \quad  & \\
                &(v, f(v)) | v \gets \text{vector  aleatorio con valores en } [0,1] \\ 
                & \text{ donde si  los valores menores que } 0.1 \text{ se transforman a } 0. \\
            \}. \quad &
        \end{align*}
        % Cálculos auxiliares 
        \State Cálculo de valores auxiliares.
        \begin{itemize}
            \item \textit{numero de cruces}  $\gets round\left(P_c \frac{M}{2}\right).$
            \item \textit{parejas a cruzar} $\gets$ vector de tamaño del \textit{numero de cruces} con valores aleatorios y únicos de enteros entre $1$ y $\frac{M}{2}$.
            \item \textit{índices cruce } $\gets \{ (2 i-2, 2 i -1 ) | i \in \textit{ parejas a cruzar } \}$. 
            \item \textit{cantidad de cromosomas a mutar }  $\gets round(P_m M)$.
            \item \textit{ índices a mutar } $\gets$ vector de índices con valores entre 1 y $M$ de tamaño \textit{ la cantidad cromosomas a mutar}. 
        \end{itemize}

        \textit{evaluaciones} $\gets M$.
        % Procedemos a evolucionar 
        \While{$evaluaciones < EMFE$}
            % Selección
            \State \textbf{Paso 1: Selección}
            \State \textit{Seleccionados} $\gets$ conjunto formado tras $M$ enfrentamientos binarios binarios de \textit{Generación }. 
            % Cruce 
            \State \textbf{Paso 2: Cruce }
            \For{ $(i,j) \in$\textit{índices cruce }  }
            \begin{align*}
                & h_1, h_2 \gets \textit{Función- cruce}(Seleccionados[i], Seleccionados[j]) \\
                & Seleccionados[i] \gets h_1 \\
                & Seleccionados[j] \gets h_2 
            \end{align*}
            \EndFor
            \State $Seleccionados \gets Seleccionados \cup$ índices parejas que no se cruzan. 

            % Mutación 
            \State \textbf{Paso 3:  Mutación}
            \For{ $i \in$ \textit{ índices a mutar }  }
            \begin{align*}
                Seleccionados[i] \gets \textit{Función-mutación}(Seleccionados[i]).
            \end{align*}
            \EndFor
        
            % Reemplazo 
            \State \textbf{Paso 4:  Reemplazo}
            \Comment{Toda la generación es reemplazada}
            \begin{equation*}
                \textit{Generación } \gets 
                \{
                    (s, F(s)) | s \in  Seleccionados
                \}.
            \end{equation*}
            % Actualizamos valores
            \State $evaluaciones \gets evaluaciones + M$ \Comment{Número evaluaciones totales función evaluación }
        \EndWhile
        % Para cada uno de los vectores          
       \State \textbf{return} \textit{Generación }.
    \end{algorithmic}
  \end{algorithm}

TODO: 
  Notemos que falta por explicar el algoritmo de torneo binario 
  y como función de mutación se mutación se ha usado la de general vecinos de la práctica anterior TODO:  añadir referencia. 

\end{document}